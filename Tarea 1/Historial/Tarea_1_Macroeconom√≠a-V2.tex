% Options for packages loaded elsewhere
\PassOptionsToPackage{unicode}{hyperref}
\PassOptionsToPackage{hyphens}{url}
%
\documentclass[
]{article}
\usepackage{amsmath,amssymb}
\usepackage{lmodern}
\usepackage{ifxetex,ifluatex}
\ifnum 0\ifxetex 1\fi\ifluatex 1\fi=0 % if pdftex
  \usepackage[T1]{fontenc}
  \usepackage[utf8]{inputenc}
  \usepackage{textcomp} % provide euro and other symbols
\else % if luatex or xetex
  \usepackage{unicode-math}
  \defaultfontfeatures{Scale=MatchLowercase}
  \defaultfontfeatures[\rmfamily]{Ligatures=TeX,Scale=1}
\fi
% Use upquote if available, for straight quotes in verbatim environments
\IfFileExists{upquote.sty}{\usepackage{upquote}}{}
\IfFileExists{microtype.sty}{% use microtype if available
  \usepackage[]{microtype}
  \UseMicrotypeSet[protrusion]{basicmath} % disable protrusion for tt fonts
}{}
\makeatletter
\@ifundefined{KOMAClassName}{% if non-KOMA class
  \IfFileExists{parskip.sty}{%
    \usepackage{parskip}
  }{% else
    \setlength{\parindent}{0pt}
    \setlength{\parskip}{6pt plus 2pt minus 1pt}}
}{% if KOMA class
  \KOMAoptions{parskip=half}}
\makeatother
\usepackage{xcolor}
\IfFileExists{xurl.sty}{\usepackage{xurl}}{} % add URL line breaks if available
\IfFileExists{bookmark.sty}{\usepackage{bookmark}}{\usepackage{hyperref}}
\hypersetup{
  pdfauthor={Vanessa Ayma Huaman; Leobardo Enríquez Hernández; Marco Méndez Atienza; Flor Yurivia Valdés de la Torre},
  hidelinks,
  pdfcreator={LaTeX via pandoc}}
\urlstyle{same} % disable monospaced font for URLs
\usepackage[margin=1in]{geometry}
\usepackage{longtable,booktabs,array}
\usepackage{calc} % for calculating minipage widths
% Correct order of tables after \paragraph or \subparagraph
\usepackage{etoolbox}
\makeatletter
\patchcmd\longtable{\par}{\if@noskipsec\mbox{}\fi\par}{}{}
\makeatother
% Allow footnotes in longtable head/foot
\IfFileExists{footnotehyper.sty}{\usepackage{footnotehyper}}{\usepackage{footnote}}
\makesavenoteenv{longtable}
\usepackage{graphicx}
\makeatletter
\def\maxwidth{\ifdim\Gin@nat@width>\linewidth\linewidth\else\Gin@nat@width\fi}
\def\maxheight{\ifdim\Gin@nat@height>\textheight\textheight\else\Gin@nat@height\fi}
\makeatother
% Scale images if necessary, so that they will not overflow the page
% margins by default, and it is still possible to overwrite the defaults
% using explicit options in \includegraphics[width, height, ...]{}
\setkeys{Gin}{width=\maxwidth,height=\maxheight,keepaspectratio}
% Set default figure placement to htbp
\makeatletter
\def\fps@figure{htbp}
\makeatother
\setlength{\emergencystretch}{3em} % prevent overfull lines
\providecommand{\tightlist}{%
  \setlength{\itemsep}{0pt}\setlength{\parskip}{0pt}}
\setcounter{secnumdepth}{-\maxdimen} % remove section numbering
\usepackage[utf8]{inputenc}
\usepackage[spanish]{babel}
\usepackage{graphicx}
\usepackage{multirow,rotating}
\pagenumbering{gobble}
\usepackage{booktabs}
\usepackage{longtable}
\usepackage{array}
\usepackage{multirow}
\usepackage{wrapfig}
\usepackage{float}
\usepackage{colortbl}
\usepackage{pdflscape}
\usepackage{tabu}
\usepackage{threeparttable}
\usepackage{threeparttablex}
\usepackage[normalem]{ulem}
\usepackage{makecell}
\usepackage{xcolor}
\ifluatex
  \usepackage{selnolig}  % disable illegal ligatures
\fi

\title{\textbf{Tarea 1, Macroeconomía II}\\
Maestría en Economía, El Colegio de México, enero - mayo, 2021}
\author{Vanessa Ayma Huaman \and Leobardo Enríquez Hernández \and Marco
Méndez Atienza \and Flor Yurivia Valdés de la Torre}
\date{06 de febrero de 2021}

\begin{document}
\maketitle

\newpage{}

{
\setcounter{tocdepth}{3}
\tableofcontents
}
\pagenumbering{gobble}
\pagenumbering{arabic}

\newpage

\hypertarget{instrucciones}{%
\section{Instrucciones}\label{instrucciones}}

\begin{enumerate}
\item Resuelva los ejercicios 8.1, 8.2, 8.4, 8.5 y 8.6. Realice estos con ayuda de su laboratorista y entregue las soluciones a máquina, utilizando LaTeX. [3 horas,1 punto cada inciso]
\item Simule una variedad de agentes que tienen ingresos permamentes diferentes e ingresos transitorios diferentes y calcule la relación entre consumo e ingreso que resulta dada una variedad de supuestos para las varianzas de cada tipo de ingreso sigiendo estos pasos:[2 horas, 1 punto cada inciso]
\begin{enumerate}
\item Cree un vector de 20 ingresos permanentes aleatorios $Y_i^P$, distribuidos normalmente, con media 10 y varianza $\sigma^P$. Cree 20 vectores (cada uno de estos vectores representa una persona) cada uno con 100 observaciones idénticas del ingreso permanente. Grafíquelos (eje x, persona; eje y, ingreso permanente).
\item Cree 20 vectores de 100 ingresos transitorios aleatorios $Y_{i,t}^T$, distribuidos normalmente, con media 0 y con varianza $\sigma^T$. Grafíquelos.
\item Cree 20 vectores de 100 ingresos totales $Y_{i,t}$, sumando el ingreso transitorio y el permanente. Grafíquelos.
\item Cree 20 vectores de 100 errores de medición $\epsilon_{i,t}$, distribuidos normalmente, con media 0 y varianza $\sigma^\epsilon>0$. Grafíquelos.
\item Cree  20 vectores de 100 consumos $C_{i,t}$ cada uno, de acuerdo a la siguiente regla $C_{i,t}=Y_i^P+0.1Y_{i,t}^T+\epsilon_{i,t}$. Grafíquelos.
\item Estime la relación lineal entre ingreso total y consumo $C_{i,t}=\alpha+\beta Y_{i,t}+\epsilon_{i,t}$. Describa el resultado de su estimación y grafíque la relación entre las observaciones del consumo y las del ingreso.
\item Incremente la varianza del ingreso permanente, y disminuya la varianza del ingreso transitorio y vuelva a estimar y graficar la relación entre el consumo y el ingreso.
\item Disminuya la varianza del ingreso permanente, y aumente la varianza del ingreso transitorio y vuelva a estimar y graficar la relación entre el consumo y el ingreso.
\end{enumerate}

\item Estudie el consumo agregado en México siguiendo estos pasos: [3 horas, 0.5 puntos cada inciso]
\begin{enumerate}
\item Obtenga, del Inegi, datos de ``C'', el consumo agregado en México, de ``Y'', el producto agregado, de ``I'', la inversión agregada, de ``G'', el gasto del gobierno y de , de ``NX'', las exportaciones netas,  entre 1980 y el tercer trimestre de 2019, EN TÉRMINOS REALES.
\item Grafíque dichas serie de tiempo juntas para compraralas visualmente.  (Compare la gráfica de las variables (de las que son siempre positivas) en su valor real original, y después de sacarles el logaritmo (cualquier logaritmo, no hace diferencia...)).
\item Grafique también la tasa de crecimiento, $ \% \Delta a_t = (a_t-a_{t-1})/a_{t-1}$,  de todas estas series. 
\item Enfóquese ahora nada más al consumo y al producto agregado. Grafique la relación entre una serie y la otra, es decir, grafique los puntos $(\% \Delta Y_t,\% \Delta C_t)$ poniendo el consumo en las ordenadas.
\item Calcule la volatilidad de ambas series de tasas de crecimiento.
\item Estime cuatro modelos lineales: $C_t=a+bY_t+\epsilon_t$, $\Delta\%C_t=a+b \Delta\%Y_t+\epsilon_t$, $\Delta\%C_t=a+b \Delta\%Y_{t-1}+\epsilon_t$ y $c_t=a+by_t+\epsilon_t$, donde las minúsculas reflejan el logaritmo de la variable en mayúscula, y reporte los valores estimados de los coeficientes, los estadísticos T, las R cuadradas, etc.
\item Explique qué se puede concluir a cerca de la Hipótesis de Ingreso Permanente para México a partir de los coeficientes encontrados.
\end{enumerate}
\item Estudie el consumo de los individuos en México, siguiendo estos pasos:[1 hora, 0.5 puntos cada inciso]
\begin{enumerate}
\item Baje los datos de un año de la ENIGH del sitio del INEGI, (Grupo 1-2018, Grupo 2-2016, etc.)  y establezca el número de hogares y el ingreso y el gasto promedio.
\item Estime una relación entre ingreso y gasto y reporte sus resultados.
\item Estime una relación entre ingreso y gasto pero para hogares unipersonales de edad entre 30 y 40 años de edad de la Ciudad de México.
\item Interprete sus resultados.
\item Para todos los hogares unipersonales, estime el valor promedio del ingreso por edad, separando la muestra en grupos de edad de cinco años cada uno y grafiquelo.
\end{enumerate}
\item Estudie el ``acertijo del premio al riesgo'' para el caso de Mexico siguiendo estos pasos: [3 horas, 0.5 puntos cada inciso]
\begin{enumerate}
\item Consiga los valores anuales de IPC, el Indice de Precios y Cotizaciones de la Bolsa Mexicana de Valores por lo menos desde 1990.
\item Calcule su tasa de retorno nominal para cada año.
\item Consiga los valores promedio anual de la tasa de interés de CETES a 7 días, o la TIIE, la tasa interbancaria de equilibrio, y de la tasa de interés a un año, para el periodo que esté disponible.
\item Calcule la diferencia entre el retorno del IPC y el retorno de invertir en CETES a distintos plazos.
\item Calcule la covarianza entre dicha diferencias y  la tasa de crecimiento real del consumo agregado de la economía mexicana. 
\item Calcule el valor de aversión relativa al riesgo que implican estos números, dado el supuesto de una utilidad con forma ARRC.
\item Ahora calcule la covarianza entre dicha diferencias y  la tasa de crecimiento real del consumo agregado DE BIENES IMPORTADOS [aquí hay una serie: www.inegi.org.mx/temas/imcp/] de la economía mexicana. 
\item Calcule el valor de aversión relativa al riesgo que implican estos números, dado el supuesto de una utilidad con forma ARRC.
\end{enumerate}

\item Utilice el método del árbol binomal para explicar el precio P=80 de un activo y valuar un ``call'' sobre él, con precio de ejercicio K=P-N donde N es el número de su equipo, asumiendo una tasa de interés de 5 por ciento: [1 horas, 0.5 puntos cada inciso]
\end{enumerate}
\newpage

\hypertarget{soluciones}{%
\section{Soluciones}\label{soluciones}}

\hypertarget{ejercicio-1}{%
\subsection{Ejercicio 1}\label{ejercicio-1}}

\textbf{Resuelva los ejercicios 8.1, 8.2, 8.4, 8.5 y 8.6 de Romer.
Realice estos con ayuda de su laboratorista y entregue las soluciones a
máquina, utilizando LaTeX. {[}3 horas,1 punto cada inciso{]}.}

\hypertarget{problema-8.1.}{%
\subsubsection{Problema 8.1.}\label{problema-8.1.}}

\textbf{Ahorro en el ciclo vital (Modigliani y Brumberg, 1954). Suponga
un individuo que vive de 0 a T y cuya utilidad vital viene dada por
\(U = \int_{t=0}^{T} u(C(t))dt\), donde \(u'(.)>0\) y \(u''(.)<0\). La
renta de este individuo es igual a \(Y_0 + gt\) cuando \(0\leq t<R\) e
igual a 0 cuando \(R\leq t \leq T\). La edad de jubilación, R, satisface
que \(0<R<T\). El tipo de interés es cero, el individuo no dispone de
ninguna riqueza inicial y no hay incertidumbre.}

\textbf{a) ¿Cuál es la restricción presupuestaria vital de este
individuo?}

El modelo planteado por Modigliani y Brumberg explica que los individuos
ahorran durante la etapa de generación de ingresos, es decir cuando
\(0\leq t<R\), gastando menos de los ingresos que genera y pensando en
su etapa de jubilación, ya que en ese periodo no generarán ingresos los
individuos. La teoría planteada se basa en la gestión y planificación de
ahorro para su jubilación, y en incentivar al ahorro a los individuos.

Con este resumen, podemos plantear la restricción presupuestaria vital a
la que se enfrenta el individuo. Para ello, consideramos los siguientes
supuestos, de acuerdo al enunciado, el tipo de interés es cero, el
individuo no dispone de ninguna riqueza inicial y no hay incertidumbre.

El valor actual del consumo de por vida debe ser menor o igual al valor
presente de por vida de los ingresos (el individuo no tiene riqueza
inicial). Así tenemos:

\setcounter{equation}{0}

\begin{equation}
\int_{t=0}^{T} C(t)dt\leqslant\int_{t=0}^{T}Y(t)dt
\end{equation}

El ingreso del individuo es \(Y(t)=Y_0+gt\) para \(0\leq t<R\) y
\(Y(t)=0\) para \(R\leq t \leq T\), el valor presente del ingreso del
ciclo de vida es

\begin{equation}
\int_{t=0}^{T}Y(t)dt=\int_{t=0}^{R}(Y_0+gt)dt
\end{equation}

Resolviendo esta integral, obtenemos como resultado:

\begin{equation}
\int_{t=0}^{R}(Y_0+gt)dt=\left|Y_0t+\frac{1}{2}gt^2\right|_{0}^{R}
\end{equation}

Simplificando:

\begin{equation}
\int_{t=0}^{R}(Y_0+gt)dt=RY_0+\frac{1}{2}gR^2
\end{equation}

Sustituyendo la ecuación (4) en la ecuación (1) se obtiene la
restricción presupuestaria del ciclo de vida:

\begin{equation}
\int_{t=0}^{T} C(t)dt\leqslant RY_0+\frac{1}{2}gR^2
\end{equation}

\textbf{b) ¿Qué trayectoria del consumo, C(t), maximiza su utilidad?}

Dado que \(u''(.)<0\) y la tasa de interés y la tasa de descuento son
iguales a cero, esto implica que la utilidad marginal del individuo es
constante y depende únicamente del consumo. La trayectoria del consumo
debe ser constante durante el periodo de vida del individuo. Entonces,
la restricción presupuestaria implica que el consumo en cada momento es
igual a los recursos de por vida divididos por la duración de la vida
(T).

En la parte (a), los recursos de por vida vienen dados por
\(RY_0+\frac{1}{2}gR^2\) y así el nivel de consumo constante es:

\begin{equation}
\bar C= \frac {1}{T} (RY_0+\frac{1}{2}gR^2)
\end{equation}

Simplificando la ecuación:

\begin{equation}
\bar C= \frac {R}{T} (Y_0+\frac{1}{2}gR)
\end{equation}

\textbf{c) ¿Qué trayectoria sigue la riqueza de este individuo en
función de t?}

La riqueza del individuo en cualquier momento t es la suma del ahorro
desde el momento 0 hasta el momento t.

\begin{equation}
W(t)=\int_{\gamma=0}^{t}S(\gamma)d\gamma
\end{equation}

Donde W es la riqueza y S es el ahorro. El ahorro del individuo en el
periodo t es la diferencia entre la renta y el consumo.

\begin{equation}
S(t)=Y(t)-C(t)
\end{equation}

Si diferenciamos los dos periodos:

En el periodo \(0\leq t<R\), el individuo logra ahorrar, siendo la
trayectoria del ahorro:

\begin{equation}
S_t=Y_t-C_t=Y_0+gt-\frac {R}{T} (Y_0+\frac{1}{2}gR)=Y_0+gt-\bar C
\end{equation}

En el periodo \(R\leq t \leq T\), el individuo no obtiene ingresos, por
lo que el individuo utiliza el ahorro que genero. La trayectoria del
ahorro en este periodo es:

\begin{equation}
S_t=Y_t-C_t=0-\frac {R}{T} (Y_0+\frac{1}{2}gR)=-\bar C
\end{equation}

Entonces, el ahorro se define:

\begin{equation}
S(t)=\left\{ \begin{array}{lcc}
              Y_0+gt-\frac {R}{T} (Y_0+\frac{1}{2}gR) &   si  & 0\leq t<R \\
            \\ -\frac {R}{T} (Y_0+\frac{1}{2}gR) &  si  & R\leq t \leq T
             \end{array}
   \right.
\end{equation}

Ahora, la trayectoria de la riqueza es:

En el periodo \(0\leq t<R\), la riqueza viene dada por:

\begin{equation}
W(t)=\int_{\gamma=0}^{t}S(\gamma)d\gamma=\int_{\gamma=0}^{t}(Y_0+g\gamma-\bar C) d\gamma
\end{equation}

Resolviendo la integral, se obtiene:

\begin{equation}
W(t)=\left|Y_0\gamma+\frac{1}{2}g\gamma^2-\bar C\gamma\right|_{\gamma=0}^{t}
\end{equation}

\begin{equation}
W(t)=Y_0t+\frac{1}{2}gt^2-\bar Ct
\end{equation}

Simplificando la ecuación, se obtiene lo siguiente:

\begin{equation}
W(t)=t(Y_0+\frac{1}{2}gt-\bar C)
\end{equation}

Para el periodo \(R\leq t \leq T\), la riqueza es igual a:

\begin{equation}
W(t)=\int_{\gamma=0}^{t}S(\gamma)d\gamma+W(R)
\end{equation}

donde W(R) es la riqueza en el momento en que el individuo se jubila.
Podemos sustituir t=R en la ecuación (15) para determinar la riqueza al
jubilarse. Esto nos da:

\begin{equation}
W(R)=R(Y_0+\frac{1}{2}gR-\bar C)
\end{equation}

Como \(\bar C= \frac {R}{T} (Y_0+\frac{1}{2}gR)\), reemplazando en la
ecuación (15), se obtiene:

\begin{equation}
W(R)=R(\frac{T}{R}\bar C-\bar C)
\end{equation}

que se simplifica a:

\begin{equation}
W(R)=(T-R)\bar C
\end{equation}

La ecuación (20) es intuitiva, dado que el individuo no recibe ingresos
al jubilarse, el individuo para poder consumir tiene que utilizar la
riqueza acumulada durante el periodo que percibió ingresos. Dado que
\((T-R)\) es el tiempo que pasa en la jubilación y dado que el individuo
consume C, la riqueza en la jubilación debe ser igual \((T-R)\bar C\).

Sustituyendo la ecuación (20) y dado que \(S_t=-\bar C\) para
\(R\leq t \leq T\) en la ecuación (17) se obtiene:

\begin{equation}
W(t)=(T-R)\bar C-\int_{\gamma=R}^{t}\bar Cd\gamma
\end{equation}

Resolviendo la integral en la ecuación (21), se obtiene lo siguiente:

\begin{equation}
W(t)=(T-R)\bar C-\bar C(t-R)
\end{equation}

Simplificando:

\begin{equation}
W(t)=(T-t)\bar C
\end{equation}

Por lo tanto, el individuo comienza a ahorrar de manera positiva una vez
que los ingresos actuales exceden los ingresos medios de por vida. La
riqueza es maximizada en la jubilación, después de lo cual la riqueza se
reduce para financiar el consumo hasta el final de la vida. El patrón
implícito en nuestro análisis se muestra en las siguientes figuras:

\includegraphics{Tarea_1_Macroeconomía-V2_files/figure-latex/FIGURA 1-1.pdf}

La figura de la izquierda muestra el ingreso y el consumo como funciones
del tiempo, asumiendo que el ingreso en el tiempo 0 supera el nivel
constante de consumo. La línea en negrita muestra ingresos que equivalen
a \(Y_0 + g\)t hasta jubilación y es igual a 0 a partir de entonces. El
consumo es constante en el nivel \(\bar C\).

La figura de la derecha representa la riqueza en función del tiempo. La
pendiente de la curva de riqueza es igual a ahorro que a su vez es igual
a la diferencia entre ingreso y consumo en la figura de la izquierda.
Riqueza aumenta (a un ritmo creciente) durante la vida laboral, ya que
los ingresos superan el consumo; la riqueza alcanza un máximo de
\((T-R)\bar C\) barrita al jubilarse. Durante la jubilación - entre
período R y T- la riqueza disminuye a una tasa constante hasta que llega
a cero al final de la vida. Dada la forma de la función de la riqueza,
este patrón de acumulación de riqueza durante el ciclo de vida se conoce
como \emph{hump saving}.

\hypertarget{problema-8.2.}{%
\subsubsection{Problema 8.2.}\label{problema-8.2.}}

\textbf{El Ingreso promedio de los agricultores es menor al ingreso
promedio de los no agricultores, pero fluctúa más año con año. Dado
esto, ¿cómo la Hipótesis del Ingreso Permanente predice que las
funciones de consumo estimado entre ambos grupos difieren?}

Sabemos que, en promedio, el ingreso transitorio es igual a cero y que
el ingreso promedio puede ser interpretado como el ingreso permanente
promedio. Así, el problema indica que el ingreso permanente de los
agricultores es menor al de los no agricultores, esto es:

\setcounter{equation}{0}

\begin{equation}
\bar{Y}_{A}^P < \bar{Y}_{NA}^P 
\end{equation}

Es decir, el hecho de que el ingreso de los agricultores fluctúe más año
con año implica que la varianza del ingreso transitorio de los
agricultores es mayor a la de los no agricultores:
\(Var(Y^T_{A}) > Var(Y_{NA}^T)\).

Considere el siguiente modelo de regresión:

\begin{equation}
C_{i} = a + bY_{i} + e_{i}
\end{equation}

Donde \(C_{i}\) es el consumo actual y, de acuerdo a la HIP, determinado
por completo por \(Y^P\), tal que \(C = Y^P\). Además, \(Y_{i}\) es el
ingreso actual, que es la suma del ingreso permanente y el transitorio,
tal que \(Y = Y^P + Y^T\).

Sabemos que el estimador de b bajo Mínimos Cuadrados Ordinarios (MCO)
tiene la forma:

\begin{equation}
\hat{b} = \frac{Var(Y^P)}{Var(Y^P) + Var(Y^T)}
\end{equation}

\(Var(Y_{A}^T)>Var(Y_{NA}^T)\) implica que \textbf{el coeficiente
estimado \(\hat{b}\) de la pendiente es menor para los agricultores que
para los no agricultores}. Esto significa que el impacto estimado de un
incremento marginal en el ingreso actual sobre el consumo es más pequeño
en el caso de los agricultores. De acuerdo a la HIP, esto se debe a que
el incremento es mucho más probable de provenir del ingreso transitorio
para los agricultores.

Por otra parte, el estimador MCO para el término constante toma la
forma:

\begin{equation}
\hat{a} = (1-\hat{b})\bar{Y}^P
\end{equation}

Los agricultores, en promedio, tienen un ingreso permanente menor a los
no agricultores. Sin embargo, como se mencionó, el estimador \(\hat{b}\)
también es menor para los agricultores, por lo que \textbf{el efecto
sobre el estimador \(\hat{a}\) es ambiguo.}

\hypertarget{problema-8.4}{%
\subsubsection{Problema 8.4}\label{problema-8.4}}

\textbf{En el modelo de la Sección 8.2, la incertidumbre sobre el
ingreso futuro no afecta al consumo. ¿Significa esto que la
incertidumbre no afecta la utilidad vitalicia esperada?}

Sabemos que la utilidad esperada vitalicia esperada es:

\setcounter{equation}{0}

\begin{equation}
E_{1}[U] = E_{1}[\sum_{t=1}^{T} (C_{t} - \frac{a}{2}C_{t}^2)]
\end{equation}

donde \(a > 0\). Esto puede ser reescrito como:

\begin{equation}
E_{1}[U] = \sum_{t=1}^{T}(E_{1}[C_{t}] - \frac{a}{2}E_{1}[C_{t}^{2}])
\end{equation}

Dado que el valor esperado del consumo en todos los periodos es
\(C_{1}\), esto es:

\begin{equation}
E_{1}[C_{t}]=C_{1}
\end{equation}

Que puede escribirse:

\begin{equation}
C_{t} = C_{1} + e_{t}
\end{equation}

donde \(E_{1}[e_{t}] = 0\) y \(Var(e_{t} = \sigma_{e_{t}}^2)\). La
ecuación (4) se cumple para todos los periodos; entonces, sustituyéndola
en la ecuación (2):

\begin{equation}
E_{1}[U] = \sum_{t=1}^T(E_{1}[C_{1} + e_{t}]-\frac{a}{2}E_{1}[(C_{1} + e_{t})^2]
\end{equation}

Como \(E_{1}[C_{1}] = C_{1}\) y \(E_{1}[e_{t}] = 0\):

\begin{equation}
E_{1}[U] = \sum_{t=1}^{T}(C_{1} - \frac{a}{2}C_{1}^{2} - \frac{a}{2}E_{1}[e_{t}^{2}])
\end{equation}

Como \(E_{1}[e_{t}^{2}] = Var(e_{t}) = \sigma_{e_{t}}^{2}\), la ecuación
(6) puede ser escrita:

\begin{equation}
E_{1}[U] = \sum_{t=1}^{T}(C_{1} - \frac{a}{2}C_{1}^{2} - \frac{a}{2}\sigma_{e_{t}}^{2})
\end{equation}

Si \(C_{t} = C_{1}\) con seguridad, tal que \(e_{t} = 0\) y
\(Var(e_{t}) = \sigma_{e_{t}}^{2} = 0\), la utilidad vitalicia es:

\begin{equation}
U = \sum_{t=1}^{T}(C_{1} - \frac{a}{2}C_{1}^{2})
\end{equation}

Es decir, se comparan las ecuaciones con incertidumbre (7) y con
certidumbre (8): como \(C_{1}\) es el mismo con o sin incertidumbre,
\textbf{la utilidad bajo incertidumbre (siempre que
\(Var(e_{t}) = \sigma_{e_{t}}^{2} > 0\)) será menor.}

\hypertarget{problema-8.5}{%
\subsubsection{Problema 8.5}\label{problema-8.5}}

\textbf{(Seguimos en este problema a Hansen y Singleton, 1983.) Suponga
que la función de utilidad instantánea adopta la forma de aversión
constante al riesgo relativo,
\(u(C_t)=\frac{C_t^{1-\theta}}{(1-\theta)} \ , \ \theta>0\) . Suponga
también que el tipo de interés real, \(r\), es constante, pero no
necesariamente igual la tasa de descuento, \(\rho\).}

\textbf{a) Halle la ecuación de Euler que relaciona \(C_t\) con las
expectativas sobre \(C_{t+1}\).}

Sabemos que la ecuación de Euler que relaciona \(C_t\) con \(C_{t+1}\)
en situaciones de certidumbre, admitiendo un tipo de interés distinto de
cero (y con las demás características idénticas al planteado en este
ejercicio, de acuerdo al texto sección 8.4 y a lo visto en clase) es:

\setcounter{equation}{0}

\begin{equation}
\frac{C_{t+1}}{C_t}=[\frac{1+r}{1+\rho}]^\frac{1}{\theta}
\end{equation}

Haremos un proceso similar, pero ahora tomando en cuenta que hay
incertidumbre.

Derivado la función de utilidad instantánea CRRA
\([u(C_t)=\frac{C_t^{1-\theta}}{(1-\theta)} \ , \ \theta>0]\) c.r.a.
\(C_t\) y \(C_{t+1}\) obtenemos la utilidad marginal del consumo para
los periodos \(t\) y \(t+1\) respectivamente:

\begin{equation} 
u'(C_t) = C_t^{-\theta} \ \ \ \ \ \ y \ \ \ \ \ \ u'(C_{t+1}) = C_{t+1}^{-\theta}
\end{equation}

Si consideramos el experimento habitual de tomar una disminución en el
consumo en una cantidad pequeña (formalmente, infinitesimal) de \(dC\)
en el período \(t\) se tiene que, dicho cambio tiene un costo de
utilidad puesto que disminuye la utilidad en dicho periodo, el cual es
igual a:

\begin{equation}
Costo \ de \ utilidad=C_t^{-\theta}dC
\end{equation}

Sin embargo, esta disminución del consumo en el periodo \(t\) provoca
que el individuo espere consumir un adicional de \((1+r)dC\) en el
periodo \(t+1\), dónde \(r\) es la tasa de interés real. Es decir, se
tiene un beneficio de utilidad esperada, el cual vamos a expresar en
términos del valor en el periodo \(t\), y no en términos del valor en el
periodo \(t+1\), y para ello usamos la tasa de descuento \(\rho\),
obteniendo:

\begin{equation}
beneficio \ de\  utilidad \ esperada = \frac{1}{1+\rho}E_t[C_{t+1}^{-\theta}(1+r)dC]
\end{equation}

Si el individuo está optimizando, un cambio marginal de este tipo no
afecta la utilidad esperada. Esto significa que el costo de la utilidad
debe ser igual al beneficio esperado de la utilidad, o:

\begin{equation}
C_t^{-\theta}dC = \frac{1}{1+\rho}E_t[C_{t+1}^{-\theta}(1+r)dC]
\end{equation}

\begin{equation}
C_{t}^{-\theta}=\frac{1+r}{1+\rho}E_t[C_{t+1}^{-\theta}]
\end{equation}

donde hemos cancelado los dD. La ecuación (6) es la ecuación de Euler.

\textbf{b) Suponga que la distribución del logaritmo de la renta y, por
tanto, la del logaritmo de \(C_{t+1}\) es normal. Llamemos \(\sigma^2\)
a la varianza de este último condicionada a la información disponible en
el período \(t\). Reescriba la expresión hallada en a) en términos de
\(lnC_1\), \(E_t [lnC_{t+1} ]\), \(\sigma^2\) y los parámetros
\(r,\rho \ y \ \theta\). {[}Pista: si una variable \(x\) está
distribuida normalmente con media \(\mu\) y varianza \(V\),
\(E[e^x]=e^{\mu} e^{V/2}\) {]}.}

Aplicando logaritmos a ambos lados de la ecuación (6) \begin{equation}
ln(C_{t}^{-\theta})=ln(\frac{1+r}{1+\rho}E_t[C_{t+1}^{-\theta}])
\end{equation}

\begin{equation}
ln(C_{t}^{-\theta})=ln(1+r)-ln(1+\rho)+ln[E_t(C_{t+1}^{-\theta})]
\end{equation}

Para cualquier variable \(x , e^{lnx}=x\), así que podemos escribir:

\begin{equation}
E_t[C_{t+1}^{-\theta}]=E_t[e^{-\theta lnC_{t+1}}]
\end{equation}

Usando la pista en la pregunta - si \(x \sim N(\mu,V)\) entonces
\(E[e^x ]=e^\mu e^{V/2}\) - entonces dado que el logaritmo del consumo
se distribuye normalmente, tenemos:

\begin{equation}
E_t[C_{t+1}^{-\theta}] = E_t[e^{-\theta E_t ln C_{t+1}}e^{\theta^2\sigma^2}]=e^{-\theta E_t ln C_{t+1}}e^{\theta^2\sigma^2}
\end{equation}

En la primera línea, hemos utilizado el hecho de que, condicional a la
información del tiempo \(t\), la varianza del logaritmo del consumo es
\(\sigma^2\). Además, hemos escrito la media del logaritmo del consumo
en el período \(t +1\), condicionado a la información del tiempo \(t\),
como \(E_t lnC_{t+1}\). Finalmente, en la última línea hemos utilizado
el hecho de que \(e^{-\theta E_t ln C_{t+1}}e^{\theta^2\sigma^2}\) es
simplemente una constante.

Sustituyendo la ecuación (10) nuevamente en la ecuación (8) tenemos:

\begin{equation}
ln(C_{t}^{-\theta})=ln(1+r)-ln(1+\rho)+ln[e^{-\theta E_t ln C_{t+1}}e^{\theta^2\sigma^2}]
\end{equation}

\begin{equation}
-\theta ln C_t = ln(1+r) - ln(1+\rho) -\theta E_t ln C_{t+1} + \theta^2 \sigma^2
\end{equation}

Dividiendo ambos lados de la ecuación (12) por \(-\theta\) , nos queda:

\begin{equation}
ln C_t = E_t ln C_{t+1} + \frac { ln(1-\rho)-ln(1+r)}{\theta - \theta \sigma^2}
\end{equation}

\textbf{c) Demuestre que si r y \(\sigma^2\) permanecen constantes a lo
largo del tiempo, el resultado de la parte b) implica que el logaritmo
del consumo sigue un paseo aleatorio cuyo rumbo es
\(lnC_{t+1}=a+lnC_t + u_{t+1}\), dónde \(u\) es ruido blanco.}

Reordenando la ecuación (13) para resolver para \(E_tlnC_{t+1}\) nos da:

\begin{equation}
E_t ln C_{t+1} =ln C_t - \frac { ln(1-\rho)-ln(1+r)}{\theta - \theta \sigma^2}
\end{equation}

\begin{equation}
E_tlnC_{t+1}=lnC_t + \frac {ln(1+r)-ln(1-\rho)}{\theta + \theta \sigma^2}
\end{equation}

La ecuación (15) implica que se espera que el consumo cambie en una
cantidad constante
\(\frac{ln(1+r)-ln(1- \rho)}{\theta + \theta \sigma^2}\) de un período
al siguiente. Los cambios en el consumo aparte de esta cantidad
determinista son impredecibles. Por la definición de expectativas
podemos escribir:

\begin{equation}
lnC_{t+1}=lnC_t+ \frac{ln(1+r)-ln(1- \rho)}{\theta + \theta \sigma^2}+U_{t+1}
\end{equation}

Dónde \(u_{t+1}\) tiene una media cero, condicionada a la información
del tiempo \(t\). Por lo tanto, el logaritmo del consumo sigue un paseo
aleatorio con deriva donde
\(\frac{ln(1+r)-ln(1- \rho)}{\theta + \theta \sigma^2}\) es el parámetro
de deriva.

\textbf{d) ¿Cómo afectan los cambios en r y en \(\sigma^2\) al
crecimiento esperado del consumo, \(E_t[ln C_{t+1}-ln C_t ]\)?
Interprete la influencia de \(\sigma^2\) sobre el crecimiento esperado
del consumo a la luz del análisis desarrollado en la Sección 7.6 sobre
el ahorro precautorio.}

De la ecuación (15), el crecimiento esperado del consumo es:

\begin{equation}
E_t[lnC_{t+1}-lnC_t]= \frac{ln(1+r)-ln(1- \rho)}{\theta + \theta \sigma^2}
\end{equation}

Para saber el efecto que provoca un cambio en \(r\) y \(\sigma^2\) en el
crecimiento esperado del consumo (\(E_t[lnC_{t+1}-lnC_t]\)) derivamos
respecto a dichas variables.

Claramente, un incremento en \(r\) incrementa el crecimiento del valor
esperado del consumo. Tenemos:

\begin{equation}
\frac{ \partial {E_t[lnC_{t+1}-lnC_t]}}{\partial r}= \frac{1}{\theta} \frac{1}{(1+r)} \ > 0
\end{equation}

Hay que tener en cuenta que cuanto menor es \(\theta\) - mayor es la
elasticidad de sustitución, \(\frac{1}{\theta}\) - más aumenta el
crecimiento del consumo debido a un aumento dado en la tasa de interés
real.

Un aumento en \(\sigma^2\) también aumenta el crecimiento del consumo:

\begin{equation}
\frac{ \partial {E_t[lnC_{t+1}-lnC_t]}}{\partial \sigma^2}= \theta > 0
\end{equation}

Es sencillo verificar que la función de utilidad CRRA tiene una tercera
derivada positiva. Como \(u'(C_t )=C_t^{-\theta }\) y
\(u''(C_t )=-\theta C_t^{-\theta }\). Entonces:

\begin{equation}
u'''(C_t )=-\theta (-\theta -1) C_t^{-\theta - 2} > 0
\end{equation}

Por tanto, un individuo con una función de utilidad CRRA exhibe el
comportamiento de ahorro preventivo explicado en la Sección 7.6. Un
aumento de la incertidumbre (medida por \(\sigma^2\), la varianza del
logaritmo del consumo) aumenta el ahorro y, por lo tanto, el crecimiento
esperado del consumo.

\hypertarget{problema-8.6}{%
\subsubsection{Problema 8.6}\label{problema-8.6}}

\textbf{Un marco para investigar el exceso de suavidad. Suponga que
\(C_t=[\frac{r}{1+r}]\{A_t+\sum^\infty_{s=0}\frac{E_t[Y_{t+s}]}{(1+r)^s}\}\)
y que \(A_{t+1}=(1+r)(A_t+Y_t-C_t)\).}

\textbf{a) Muestre que estos supuestos implican que \(E_t[C_{t+1}]=C_t\)
(y entonces que el consumo sigue una caminata aleatoria) y que
\(\sum^\infty_{s=0}\frac{E_t[C_{t+s}]}{(1+r)^s}=A_t+\sum^\infty_{s=0}\frac{E_t[Y_{t+s}]}{(1+r)^s}\).}

Sustituyendo la expresión para el consumo en el periodo t:

\setcounter{equation}{0}

\begin{equation}
C_t=[\frac{r}{1+r}]\{A_t+\sum^\infty_{s=0}\frac{E_t[Y_{t+s}]}{(1+r)^s}\}
\end{equation}

en la expresión de la riqueza en el periodo t+1:

\begin{equation}
A_{t+1}=(1+r)(A_t+Y_t-C_t)
\end{equation}

obtenemos la expresión:

\begin{equation}
A_{t+1}=(1+r)[A_t+Y_t-\frac{r}{1+r}A_t-\frac{r}{1+r}(Y_t+\frac{E_tY_{t+1}}{1+r}+\frac{E_tY_{t+2}}{(1+r)^2}+...)]
\end{equation}

Obteniendo un común denominador de \((1+r)\) y cancelando los términos
\((1+r)\) obtenemos:

\begin{equation}
A_{t+1}=A_t+Y_t-r(\frac{E_tY_{t+1}}{1+r}+\frac{E_tY_{t+2}}{(1+r)^2}+...)
\end{equation}

Como la ecuación
\(C_t=[\frac{r}{1+r}]\{A_t+\sum^\infty_{s=0}\frac{E_t[Y_{t+s}]}{(1+r)^s}\}\)
se mantiene en todos los periodos, podemos escribir el consumo en el
periodo t+1 como:

\begin{equation}
C_{t+1}=\frac{r}{1+r}[A_{t+1}+\sum^\infty_{s=0}\frac{E_{t+1}[Y_{t+1+s}]}{(1+r)^s}]
\end{equation}

Sustituyendo la ecuación (4) en (5) tenemos la ecuación:

\begin{equation}
C_{t+1}=\frac{r}{1+r}[A_t+Y_t-r(\frac{E_tY_{t+1}}{1+r}+\frac{E_tY_{t+2}}{(1+r)^2}+...)+(E_{t+1}Y_{t+1}+\frac{E_{t+1}Y_{t+2}}{1+r}+...)]
\end{equation}

Tomando la esperanza, condicional a la información en el tiempo t, de
ambos lados de esta última ecuación, tenemos:

\begin{equation}
E_tC_{t+1}=\frac{r}{1+r}[A_t+Y_t-r(\frac{E_tY_{t+1}}{1+r}+\frac{E_tY_{t+2}}{(1+r)^2}+...)+(E_{t}Y_{t+1}+\frac{E_{t}Y_{t+2}}{1+r}+...)]
\end{equation}

donde hemos usado la ley de las proyeccciones iteradas tal que para toda
variable \(x\), \(E_tE_{t+1}x_{t+2}=E_tx_{t+2}\). Si esto no se
sostuviera, los individuos estarían esperando a revisar su estimación
hacia arriba o hacia abajo, y entonces su expectativa original no podría
haber sido racional.

Y simplificando términos de la última ecuación, tenemos:

\begin{equation}
E_tC_{t+1}=\frac{r}{1+r}[A_t+Y_t-(1-\frac{r}{1+r})E_tY_{t+1}+(\frac{1}{(1+r)}-\frac{r}{(1+r)^2})E_tY_{t+2}+...]
\end{equation}

el cual se simplifica a:

\begin{equation}
E_tC_{t+1}=\frac{r}{1+r}[A_t+Y_t+\frac{E_tY_{t+1}}{1+r}+\frac{E_tY_{t+2}}{(1+r)^2}+...]
\end{equation}

Usando la notación de suma, y que \(E_tY_t=Y_t\) tenemos la expresión:

\begin{equation}
E_tC_{t+1}=\frac{r}{1+r}[A_t+\sum^\infty_{s=0}\frac{E_tY_{t+s}}{(1+r)^s}]
\end{equation}

El lado derecho de la ecuación
\[C_t=[\frac{r}{1+r}]\{A_t+\sum^\infty_{s=0}\frac{E_t[Y_{t+s}]}{(1+r)^s}\}\]
y de la ecuación
\(E_tC_{t+1}=\frac{r}{1+r}[A_t+\sum^\infty_{s=0}\frac{E_tY_{t+s}}{(1+r)^s}]\)
son iguales por lo que:

\begin{equation}
E_tC_{t+1}=C_t
\end{equation}

El consumo sigue una caminata aleatoria; cambios en el consumo son
impredecibles.

Como el consumo sigue una caminata aleatoria, el mejor estimador del
consumo para cualquier periodo futuro es simplemente el valor esperado
del consumo en ese periodo. Esto es, para toda \(s\geq 0\), podemos
escribir:

\begin{equation}
E_tC_{t+s}=C_t
\end{equation}

Usando esta ecuación \(E_tC_{t+s}=C_t\), podemos escribir el valor
presente de la trayectoria esperada de consumo como:

\begin{equation}
\sum^\infty_{s=0}\frac{E_t[C_{t+s}]}{(1+r)^s}=\sum^\infty_{s=0} \frac{C_t}{(1+r)^s}=C_t\sum^\infty_{s=0}\frac{1}{(1+r)^s}
\end{equation}

Como \(\frac{1}{1+r}<1\), la suma infinita del lado derecho de la última
ecuación converge a \(\frac{1}{[1-\frac{1}{1+r}]}=\frac{(1+r)}{r}\) y
entonces tenemos:

\begin{equation}
\sum^\infty_{s=0}\frac{E_t[C_{t+s}]}{(1+r)^s}=\frac{1+r}{r}C_t
\end{equation}

Sustituyendo la ecuación
\(C_t=[\frac{r}{1+r}]\{A_t+\sum^\infty_{s=0}\frac{E_t[Y_{t+s}]}{(1+r)^s}\}\)
por \(C_t\) en el lado derecho de la ecuación
\(\sum^\infty_{s=0}\frac{E_t[C_{t+s}]}{(1+r)^s}=\frac{1+r}{r}C_t\)
tenemos:

\begin{equation}
\sum^\infty_{s=0}\frac{E_t[C_{t+s}]}{(1+r)^s}=\frac{r}{1+r}(\frac{1+r}{r})[A_t+\sum^\infty_{s=0}\frac{E_t[Y_{t+s}]}{(1+r)^s}]=A_t+\sum^\infty_{s=0}\frac{E_t[Y_{t+s}]}{(1+r)^s}
\end{equation}

Esta última ecuación establece que el valor presente de la trayectoria
esperada del consumo iguala a la riqueza inicial más el valor presente
de la trayectoria esperada del ingreso.

\textbf{b) Suponga que \(\Delta Y_t=\phi \Delta Y_{t+1}+u_t\), donde
\emph{u} es ruido blanco. Suponga que \(Y_t\) excede \(E_{t-1}[Y_t]\) en
1 unidad (esto es, suponga \emph{u\_t=1}). ¿En cuánto incrementa el
consumo?}

Tomando el valor esperado, como del tiempo t-1, en ambos lados de la
ecuación
\(C_t=[\frac{r}{1+r}]\{A_t+\sum^\infty_{s=0}\frac{E_t[Y_{t+s}]}{(1+r)^s}\}\),
tenemos:

\begin{equation}
E_{t-1}C_t=\frac{r}{1+r}[A_t+\sum^\infty_{s=0}\frac{E_{t-1}(Y_{t+s})}{(1+r)^s}]
\end{equation}

donde hemos usado el hecho de que \(A_t=(1+r)[A_{t-1}+Y_{t-1}-C_{t-1}]\)
no es incierto como en t-1.

Adicionalmente, hemos usado la ley de las proyecciones iteradas tal que
\(E_{t-1}E_{t}[Y_{t+s}]=E_{t-1}[Y_{t+s}]\).

Restando la ecuación (16) de la ecuación
\(C_t=[\frac{r}{1+r}]\{A_t+\sum^\infty_{s=0}\frac{E_t[Y_{t+s}]}{(1+r)^s}\}\)
tenemos el cambio en el consumo:

\begin{equation}
C_t-E_{t-1}C_t=\frac{r}{1+r}[\sum^\infty_{s=0}\frac{E_t[Y_{t+s}]}{(1+r)^s}-\sum^\infty_{s=0}\frac{E_{t-1}[Y_{t+s}]}{(1+r)^s}]=\frac{r}{1+r}[\sum^\infty_{s=0}\frac{E_t[Y_{t+s}]-E_{t-1}[Y_{t+s}]}{(1+r)^s}]
\end{equation}

El cambio en el consumo será la fracción \(\frac{r}{(1+r)}\) del valor
presente del cambio en el ingreso vitalicio esperado.

El siguiente paso es determinar el valor presente del cambio en el
ingreso vitalicio esperado:

\begin{equation}
\sum^\infty_{s=0}\frac{E_t[Y_{t+s}]-E_{t-1}[Y_{t+s}]}{(1+r)^s}=[Y_t-E_{t-1}Y_t]+[\frac{E_tY_{t+1}-E_{t-1}Y_{t+1}}{1+r}]+[\frac{E_tY_{t+2}-E_{t-1}Y_{t+2}}{(1+r)^2}]+...
\end{equation}

En lo subsecuente ``se espera que sea más alto'' significa que ``el
valor esperado, como por ejemplo en el periodo t, sea mayor a lo que era
en el periodo t-1''. Como \(u_t=1\), entonces:

\begin{equation}
Y_t-E_{t-1}Y_t=1
\end{equation}

En el periodo t+1, como \(\Delta Y_{t+1}=\phi \Delta Y_t + u_{t+1}\), el
cambio en \(Y_{t+1}\) se espera que sea \(\phi \Delta Y_t=\phi\) más
alto. Entonces, el nivel de \(Y_{t+1}\) se espera que sea más alto por
\(1+\phi\). Entonces:

\begin{equation}
\frac{E_tY_{t+1}-E_{t-1}Y_{t+1}}{1+r}=\frac{1+\phi}{1+r}
\end{equation}

En el periodo t+2, como \(\Delta Y_{t+2}=\phi \Delta Y_{t+1}+u_{t+2}\),
el cambio en \(Y_{t+2}\) se espera que sea más alto por
\(\phi \Delta Y_{t+1}=\phi^2\). Entonces el nivel de \(Y_{t+2}\) se
espera que sea más alto por \(1+\phi +\phi^2\). Por lo tanto, tenemos:

\begin{equation}
\frac{E_tY_{t+2}-E_{t-1}Y_{t+2}}{(1+r)^2}=\frac{1+\phi+\phi^2}{(1+2)^2}
\end{equation}

Este comportamiento debe de ser claro. Tenemos que:

\begin{equation}
\sum^\infty_{s=0}\frac{E_t[Y_{t+s}]-E_{t-1}[Y_{t+s}]}{(1+r)^s}=1+\frac{1+\phi}{1+r}+\frac{1+\phi + \phi^2}{(1+r)^2}+\frac{1+\phi+\phi^2`+\phi^3}{(1+r)^3}+...
\end{equation}

Notemos que la serie infinita puede escribirse como:

\begin{equation}
\sum^\infty_{s=0}\frac{E_t[Y_{t+s}]-E_{t-1}[Y_{t+s}]}{(1+r)^s}=[1+\frac{1}{1+r}+\frac{1}{(1+r)^2}+...]+[\frac{\phi}{1+r}+\frac{\phi}{(1+r)^2}+\frac{\phi}{(1+r)^3}+...]+[\frac{\phi^2}{(1+r)^2}+\frac{\phi^2}{(1+r)^3}+...]+...
\end{equation}

Si definimos \(\gamma =\frac{1}{1+r}\), la primera suma del lado rerecho
de la última ecuación converge a \(\frac{1}{1-\gamma}\), la segunda suma
converge a \(\frac{\phi \gamma}{(1-\gamma)}\), la tercera suma converge
a \(\frac{\phi^2 \gamma^2}{1-\gamma}\), y así sucesivamente.

Por lo tanto la ecuación anterior puede escribirse como

\begin{equation}
\sum^\infty_{s=0}\frac{E_t[Y_{t+s}]-E_{t-1}[Y_{t+s}]}{(1+r)^s}=\frac{1}{1-\gamma}[1+\phi \gamma+\phi^2 \gamma^2+...]=\frac{1}{(1-\gamma)} \frac{1}{(1-\phi \gamma)}
\end{equation}

Reescribiendo esta última ecuación con la definición de
\(\gamma =\frac{1}{1+r}\) tenemos:

\begin{equation}
\sum^\infty_{s=0}\frac{E_t[Y_{t+s}]-E_{t-1}[Y_{t+s}]}{(1+r)^s}=\frac{1}{1-[\frac{1}{1+r}]}\frac{1}{1-[\frac{\phi}{1+r}]}=\frac{(1+r)}{r}\frac{(1+r)}{(1+r-\phi)}
\end{equation}

Sustituyendo esta última ecuación en la ecuación inmediata anterior
tenemos el siguiente cambio en el consumo:

\begin{equation}
C_t-E_{t-1}C_t=\frac{r}{(1+r)}[\frac{(1+r)}{r}\frac{(1+r)}{(1+r-\phi)}]=\frac{(1+r)}{(1+r-\phi)}
\end{equation}

\textbf{c) Para el caso de \(\phi >0\), cuál tiene una mayor varianza,
la innovación en el ingreso, \(u_t\), o la innovación en el
consumo,\(C_t-E_{t-1}[C_t]\)? ¿Utilizan los consumidores el ahorro y el
endeudamiento para suavizar la senda del consumo en relaciónal ingreso
en este modelo? Explique.}

La varianza del cambio en el consumo es:

\begin{equation}
var(C_t-E_{t-1}C_t)=var[\frac{(1+r)}{(1+r-\phi)}u_t]=[\frac{(1+r)}{(1+r-\phi)}]^2 var(u_t)>var(u_t)
\end{equation}

Como\(\frac{(1+r)}{(1+r-\phi)}>1\), la varianza del cambio en el consumo
es más grande que la varianza del cambio en el ingreso.

Intuitivamente, un cambio en el ingreso significa que en promedio, el
consumidor experimentará más cambios en el ingreso en la misma dirección
en los periodos futuros.

No es claro si el consumidor usa el ahorro o la deuda para suavizar el
consumo relativo al ingreso. El ingreso no es estacionario, entoces no
es obvio lo que significa suavizarlo.

\newpage

\hypertarget{ejercicio-2}{%
\subsection{Ejercicio 2}\label{ejercicio-2}}

\textbf{Simule una variedad de agentes que tienen ingresos permamentes
diferentes e ingresos transitorios diferentes y calcule la relación
entre consumo e ingreso que resulta dada una variedad de supuestos para
las varianzas de cada tipo de ingreso sigiendo estos pasos:{[}2 horas, 1
punto cada inciso{]}}

\hypertarget{ejercicio-2.a}{%
\subsubsection{Ejercicio 2.a}\label{ejercicio-2.a}}

\textbf{Cree un vector de 20 ingresos permanentes aleatorios \(Y_i^P\),
distribuidos normalmente, con media 10 y varianza \(\sigma^P\). Cree 20
vectores (cada uno de estos vectores representa una persona) cada uno
con 100 observaciones idénticas del ingreso permanente. Grafíquelos (eje
x, persona; eje y, ingreso permanente).}

Se creó un vector de 20 observaciones aleatorias distribuidas
normalmente con media igual a 10 y varianza de 4 que representan el
ingreso permanente de 20 individuos, esto es: \(\mu_{Y^P_{i}} = 10\) y
\(Var(Y^P_{i}) = 4\). Posteriormente, se construyó una matriz con 100
observaciones para cada individuo, siendo estas idénticas a su ingreso
permanente.

Así, la siguiente gráfica muestra el ingreso permanente de 20
individuos, donde puede observarse que la media de todas las
observaciones es bastante cercana al valor de 10, dado que
\(\mu_{Y^P_{i}} = 10\):

\includegraphics{Tarea_1_Macroeconomía-V2_files/figure-latex/GRÁFICA INGRESO PERMANENTE-1.pdf}
\newpage

\hypertarget{ejercicio-2.b}{%
\subsubsection{Ejercicio 2.b}\label{ejercicio-2.b}}

\textbf{Cree 20 vectores de 100 ingresos transitorios aleatorios
\(Y_{i,t}^T\), distribuidos normalmente, con media 0 y con varianza
\(\sigma^T\). Grafíquelos.}

Una vez creados los 20 vectores con 100 ingresos transitorios para cada
individuo con \(\mu_{Y^T_{i}} = 0\) y \(Var(Y^T_{i}) = 4\), y con el
objetivo de facilitar la representación gráfica, se presenta la primera
observación para los 20 individuos, así como 100 observaciones para el
primer individuo, donde puede observarse que ambas medias son cercanas a
cero:

\includegraphics{Tarea_1_Macroeconomía-V2_files/figure-latex/GRÁFICO INGRESO TRANSITORIO -1.pdf}
\includegraphics{Tarea_1_Macroeconomía-V2_files/figure-latex/GRÁFICO INGRESO TRANSITORIO -2.pdf}
\newpage

\hypertarget{ejercicio-2.c}{%
\subsubsection{Ejercicio 2.c}\label{ejercicio-2.c}}

\textbf{Cree 20 vectores de 100 ingresos totales \(Y_{i,t}\), sumando el
ingreso transitorio y el permanente. Grafíquelos.}

Después de crear los 20 vectores aleatorios con 100 ingresos totales con
base en la expresión \(Y = Y^P + Y^T\), a continuación de muestran dos
gráficas, una que exhibe la primera observación del ingreso total de los
20 individuos, y otra con 100 diferentes ingresos totales para el primer
individuo:

\includegraphics{Tarea_1_Macroeconomía-V2_files/figure-latex/GRÁFICA INGRESO TOTAL-1.pdf}
\includegraphics{Tarea_1_Macroeconomía-V2_files/figure-latex/GRÁFICA INGRESO TOTAL-2.pdf}

Puede observarse que los valores del ingreso total para todos los
individuos, \(Y\), la media se mantiene alrededor del valor de 10,
producto de la propiedad de linealidad de la esperanza:
\(E[Y] = E[Y^P] + E[Y^T] = 10 + 0 = 10\).

\newpage

\hypertarget{ejercicio-2.d}{%
\subsubsection{Ejercicio 2.d}\label{ejercicio-2.d}}

\textbf{Cree 20 vectores de 100 errores de medición \(\epsilon_{i,t}\),
distribuidos normalmente, con media 0 y varianza \(\sigma^\epsilon>0\).
Grafíquelos.}

Se crearon 20 vectores con 100 errores cada uno, con media cero y
varianza de 0.25, esto es: \(\mu_{e_{i}} = 0\) y \(Var(e_{i}) = 0.25\).
Así, se presenta la primera observación de errores para los 20
individuos, así como 100 errores para el individuo 1, donde ambas medias
se encuentran alrededor del cero.

\includegraphics{Tarea_1_Macroeconomía-V2_files/figure-latex/GRÁFICA ERRORES -1.pdf}
\includegraphics{Tarea_1_Macroeconomía-V2_files/figure-latex/GRÁFICA ERRORES -2.pdf}
\newpage

\hypertarget{ejercicio-2.e}{%
\subsubsection{Ejercicio 2.e}\label{ejercicio-2.e}}

\textbf{Cree 20 vectores de 100 consumos \(C_{i,t}\) cada uno, de
acuerdo a la siguiente regla
\(C_{i,t}=Y_i^P+0.1Y_{i,t}^T+\epsilon_{i,t}\). Grafíquelos.}

Una vez calculados 100 diferentes consumos para cada uno de los 20
individuos, se graficó la primera observación para todos los individuos
y todas las observaciones para el primer individuo:

\includegraphics{Tarea_1_Macroeconomía-V2_files/figure-latex/GRAFICA CONSUMO-1.pdf}
\includegraphics{Tarea_1_Macroeconomía-V2_files/figure-latex/GRAFICA CONSUMO-2.pdf}

Dado que \(C_{i,t}=Y_i^P+0.1Y_{i,t}^T+\epsilon_{i,t}\), entonces
\(E[C_{i,t}] = E[Y_i^P] + 0.1E[Y_{i,t}^T] + E[\epsilon_{i,t}] = 10 + 0.1(0) + 0 = 10\),
lo cual es consistente con las gráficas.

\newpage

\hypertarget{ejercicio-2.f}{%
\subsubsection{Ejercicio 2.f}\label{ejercicio-2.f}}

\textbf{Estime la relación lineal entre ingreso total y consumo
\(C_{i,t}=\alpha+\beta Y_{i,t}+\epsilon_{i,t}\). Describa el resultado
de su estimación y grafíque la relación entre las observaciones del
consumo y las del ingreso.}

Al realizar la regresión del modelo, puede observarse una relación
positiva entre ambas variables. Además, hay una concentración de las
observaciones más grande alrededor de los valores de 10 del Ingreso, lo
cual es consistente con el hecho de que el Ingreso Total se compone del
Ingreso Permanente (con media 10) y el Ingreso Transitorio (con media
0):

\includegraphics{Tarea_1_Macroeconomía-V2_files/figure-latex/REGRESIÓN1-1.pdf}

En las tablas siguientes, se exhiben los coeficientes estimados de las
20 regresiones realizadas, una por individuo. En ella puede observarse
que, en general, el estimador \(\hat\alpha\) suele estar entre valores
de 7 y 13, lo que indica una variación moderada entre los individuos.
Por otro lado, el estimador \(\hat\beta\), que indica la responsividad
del consumo ante el ingreso, es, como se esperaba positiva y con valores
alrededor de 0.05 y 1.5, lo cual es consistente con el hecho de que el
la aportación del ingreso transitorio a cambios en el consumo es baja en
el modelo planteado \(C_{i,t}=Y_i^P+0.1Y_{i,t}^T+\epsilon_{i,t}\).

\newpage
\begin{landscape}


\begin{table}[!htbp] \centering 
  \caption{Tabla de Regresión 1} 
  \label{} 
\footnotesize 
\begin{tabular}{@{\extracolsep{5pt}}lcccccccccc} 
\\[-1.8ex]\hline 
\hline \\[-1.8ex] 
 & \multicolumn{10}{c}{Variable Dependiente} \\ 
\cline{2-11} 
\\[-1.8ex] & \multicolumn{10}{c}{Consumo} \\ 
 & 1 & 2 & 3 & 4 & 5 & 6 & 7 & 8 & 9 & 10 \\ 
\\[-1.8ex] & (1) & (2) & (3) & (4) & (5) & (6) & (7) & (8) & (9) & (10)\\ 
\hline \\[-1.8ex] 
 Ingreso & 0.127$^{***}$ & 0.087$^{***}$ & 0.089$^{***}$ & 0.079$^{***}$ & 0.033 & 0.113$^{***}$ & 0.061$^{***}$ & 0.098$^{***}$ & 0.044$^{*}$ & 0.139$^{***}$ \\ 
  & (0.027) & (0.025) & (0.025) & (0.025) & (0.027) & (0.023) & (0.020) & (0.027) & (0.024) & (0.025) \\ 
  & & & & & & & & & & \\ 
 Constante & 9.915$^{***}$ & 8.195$^{***}$ & 9.500$^{***}$ & 10.271$^{***}$ & 7.294$^{***}$ & 8.088$^{***}$ & 7.238$^{***}$ & 6.478$^{***}$ & 7.510$^{***}$ & 8.909$^{***}$ \\ 
  & (0.309) & (0.249) & (0.262) & (0.269) & (0.204) & (0.213) & (0.161) & (0.201) & (0.199) & (0.259) \\ 
  & & & & & & & & & & \\ 
\hline \\[-1.8ex] 
Observations & 100 & 100 & 100 & 100 & 100 & 100 & 100 & 100 & 100 & 100 \\ 
R$^{2}$ & 0.188 & 0.107 & 0.115 & 0.093 & 0.016 & 0.196 & 0.085 & 0.115 & 0.034 & 0.244 \\ 
Adjusted R$^{2}$ & 0.179 & 0.098 & 0.106 & 0.084 & 0.006 & 0.188 & 0.076 & 0.106 & 0.024 & 0.236 \\ 
Residual Std. Error (df = 98) & 0.552 & 0.489 & 0.458 & 0.476 & 0.495 & 0.478 & 0.472 & 0.538 & 0.474 & 0.492 \\ 
F Statistic (df = 1; 98) & 22.645$^{***}$ & 11.781$^{***}$ & 12.769$^{***}$ & 10.058$^{***}$ & 1.549 & 23.867$^{***}$ & 9.151$^{***}$ & 12.704$^{***}$ & 3.452$^{*}$ & 31.635$^{***}$ \\ 
\hline 
\hline \\[-1.8ex] 
P-valor & \multicolumn{10}{r}{$^{*}$p$<$0.1; $^{**}$p$<$0.05; $^{***}$p$<$0.01} \\ 
\end{tabular} 
\end{table} 

\begin{table}[!htbp] \centering 
  \caption{Tabla de Regresión 1 (Continuación)} 
  \label{} 
\footnotesize 
\begin{tabular}{@{\extracolsep{5pt}}lcccccccccc} 
\\[-1.8ex]\hline 
\hline \\[-1.8ex] 
 & \multicolumn{10}{c}{Variable Dependiente} \\ 
\cline{2-11} 
\\[-1.8ex] & \multicolumn{10}{c}{Consumo} \\ 
 & 11 & 12 & 13 & 14 & 15 & 16 & 17 & 18 & 19 & 20 \\ 
\\[-1.8ex] & (1) & (2) & (3) & (4) & (5) & (6) & (7) & (8) & (9) & (10)\\ 
\hline \\[-1.8ex] 
 Ingreso & 0.095$^{***}$ & 0.087$^{***}$ & 0.083$^{***}$ & 0.139$^{***}$ & 0.148$^{***}$ & 0.089$^{***}$ & 0.111$^{***}$ & 0.090$^{***}$ & 0.098$^{***}$ & 0.111$^{***}$ \\ 
  & (0.025) & (0.023) & (0.023) & (0.027) & (0.027) & (0.029) & (0.026) & (0.026) & (0.024) & (0.022) \\ 
  & & & & & & & & & & \\ 
 Constante & 7.709$^{***}$ & 7.845$^{***}$ & 10.089$^{***}$ & 9.077$^{***}$ & 9.515$^{***}$ & 11.591$^{***}$ & 8.407$^{***}$ & 7.766$^{***}$ & 6.184$^{***}$ & 10.557$^{***}$ \\ 
  & (0.211) & (0.203) & (0.254) & (0.294) & (0.310) & (0.364) & (0.241) & (0.223) & (0.168) & (0.269) \\ 
  & & & & & & & & & & \\ 
\hline \\[-1.8ex] 
Observations & 100 & 100 & 100 & 100 & 100 & 100 & 100 & 100 & 100 & 100 \\ 
R$^{2}$ & 0.129 & 0.126 & 0.121 & 0.207 & 0.231 & 0.090 & 0.160 & 0.112 & 0.146 & 0.205 \\ 
Adjusted R$^{2}$ & 0.120 & 0.117 & 0.112 & 0.199 & 0.223 & 0.081 & 0.151 & 0.103 & 0.137 & 0.197 \\ 
Residual Std. Error (df = 98) & 0.465 & 0.467 & 0.498 & 0.564 & 0.509 & 0.522 & 0.496 & 0.488 & 0.494 & 0.462 \\ 
F Statistic (df = 1; 98) & 14.520$^{***}$ & 14.154$^{***}$ & 13.483$^{***}$ & 25.608$^{***}$ & 29.429$^{***}$ & 9.735$^{***}$ & 18.641$^{***}$ & 12.392$^{***}$ & 16.762$^{***}$ & 25.344$^{***}$ \\ 
\hline 
\hline \\[-1.8ex] 
P-valor & \multicolumn{10}{r}{$^{*}$p$<$0.1; $^{**}$p$<$0.05; $^{***}$p$<$0.01} \\ 
\end{tabular} 
\end{table} 

\end{landscape}

\newpage

\hypertarget{ejercicio-2.g}{%
\subsubsection{Ejercicio 2.g}\label{ejercicio-2.g}}

\textbf{Incremente la varianza del ingreso permanente, y disminuya la
varianza del ingreso transitorio y vuelva a estimar y graficar la
relación entre el consumo y el ingreso.}

Hasta ahora, los valores de las varianzas del ingreso permanente y el
transitorio eran, respectivamente: \(Var(Y^P) = Var(Y^T) = 4\). Ahora,
se utilizaron los valores \(Var'(Y^P) = 9\) y \(Var'(Y^T) = 0.25\). Una
vez realizado el procedimiento de código correspondiente, se encontró la
siguiente relación entre el Nuevo Consumo y el Nuevo Ingreso:

\includegraphics{Tarea_1_Macroeconomía-V2_files/figure-latex/REGRESIÓN4-1.pdf}

En la siguiente tabla se pueden observar dos características relevantes:

\begin{enumerate}
\def\labelenumi{\arabic{enumi}.}
\item
  Los parámetros estimados \(\hat{\alpha}\) para el segundo modelo
  presentan una varianza claramente mayor a la del modelo anterior, lo
  cual tiene sentido con los supuestos de este ejercicio. Esto es, la
  varianza del ingreso permanente es mayor a través de individuos.
\item
  Por otra parte, se observa que la concentración de observaciones es
  más compacta que en el caso anterior. Esto se debe a que la varianza
  del Ingreso Transitorio es más baja que antes, lo que implica una
  menor dispersión de los datos alrededor del modelo estimado. Es decir,
  la relación entre el ingreso y el consumo es mucho más consistente en
  este ejemplo y, a través de la mayoría de individuos y observaciones,
  se comporta de manera casi lineal.
\end{enumerate}

\newpage
\begin{landscape}


\begin{table}[!htbp] \centering 
  \caption{Tabla de Regresión 2} 
  \label{} 
\footnotesize 
\begin{tabular}{@{\extracolsep{5pt}}lcccccccccc} 
\\[-1.8ex]\hline 
\hline \\[-1.8ex] 
 & \multicolumn{10}{c}{Variable Dependiente} \\ 
\cline{2-11} 
\\[-1.8ex] & \multicolumn{10}{c}{Consumo} \\ 
 & 1 & 2 & 3 & 4 & 5 & 6 & 7 & 8 & 9 & 10 \\ 
\\[-1.8ex] & (1) & (2) & (3) & (4) & (5) & (6) & (7) & (8) & (9) & (10)\\ 
\hline \\[-1.8ex] 
 Ingreso & 0.372$^{***}$ & $-$0.014 & 0.113 & 0.130 & 0.179 & 0.075 & 0.110 & 0.276$^{**}$ & $-$0.008 & 0.203$^{**}$ \\ 
  & (0.107) & (0.094) & (0.093) & (0.104) & (0.119) & (0.087) & (0.095) & (0.120) & (0.084) & (0.095) \\ 
  & & & & & & & & & & \\ 
 Constante & 4.566$^{***}$ & 7.894$^{***}$ & 8.298$^{***}$ & 7.129$^{***}$ & 5.062$^{***}$ & 7.764$^{***}$ & 8.051$^{***}$ & 9.745$^{***}$ & 12.929$^{***}$ & 1.525$^{***}$ \\ 
  & (0.763) & (0.742) & (0.871) & (0.846) & (0.758) & (0.735) & (0.873) & (1.623) & (1.080) & (0.183) \\ 
  & & & & & & & & & & \\ 
\hline \\[-1.8ex] 
Observations & 100 & 100 & 100 & 100 & 100 & 100 & 100 & 100 & 100 & 100 \\ 
R$^{2}$ & 0.109 & 0.0002 & 0.015 & 0.016 & 0.023 & 0.008 & 0.013 & 0.051 & 0.0001 & 0.045 \\ 
Adjusted R$^{2}$ & 0.100 & $-$0.010 & 0.005 & 0.006 & 0.013 & $-$0.003 & 0.003 & 0.041 & $-$0.010 & 0.035 \\ 
Residual Std. Error (df = 98) & 0.538 & 0.486 & 0.458 & 0.478 & 0.510 & 0.478 & 0.481 & 0.532 & 0.483 & 0.495 \\ 
F Statistic (df = 1; 98) & 11.974$^{***}$ & 0.022 & 1.487 & 1.573 & 2.277 & 0.743 & 1.323 & 5.271$^{**}$ & 0.010 & 4.593$^{**}$ \\ 
\hline 
\hline \\[-1.8ex] 
P-valor & \multicolumn{10}{r}{$^{*}$p$<$0.1; $^{**}$p$<$0.05; $^{***}$p$<$0.01} \\ 
\end{tabular} 
\end{table} 

\begin{table}[!htbp] \centering 
  \caption{Tabla de Regresión 2 (Continuación)} 
  \label{} 
\footnotesize 
\begin{tabular}{@{\extracolsep{5pt}}lcccccccccc} 
\\[-1.8ex]\hline 
\hline \\[-1.8ex] 
 & \multicolumn{10}{c}{Variable Dependiente} \\ 
\cline{2-11} 
\\[-1.8ex] & \multicolumn{10}{c}{Consumo} \\ 
 & 11 & 12 & 13 & 14 & 15 & 16 & 17 & 18 & 19 & 20 \\ 
\\[-1.8ex] & (1) & (2) & (3) & (4) & (5) & (6) & (7) & (8) & (9) & (10)\\ 
\hline \\[-1.8ex] 
 Ingreso & $-$0.094 & 0.057 & 0.175$^{*}$ & $-$0.012 & $-$0.100 & 0.206$^{*}$ & 0.171 & 0.102 & 0.131 & 0.110 \\ 
  & (0.091) & (0.105) & (0.100) & (0.106) & (0.099) & (0.114) & (0.111) & (0.089) & (0.109) & (0.087) \\ 
  & & & & & & & & & & \\ 
 Constante & 7.899$^{***}$ & 12.178$^{***}$ & 4.243$^{***}$ & 13.462$^{***}$ & 16.341$^{***}$ & 9.124$^{***}$ & 6.966$^{***}$ & 12.878$^{***}$ & 10.652$^{***}$ & 6.760$^{***}$ \\ 
  & (0.661) & (1.345) & (0.515) & (1.405) & (1.465) & (1.304) & (0.931) & (1.278) & (1.334) & (0.661) \\ 
  & & & & & & & & & & \\ 
\hline \\[-1.8ex] 
Observations & 100 & 100 & 100 & 100 & 100 & 100 & 100 & 100 & 100 & 100 \\ 
R$^{2}$ & 0.011 & 0.003 & 0.030 & 0.0001 & 0.010 & 0.032 & 0.024 & 0.013 & 0.015 & 0.016 \\ 
Adjusted R$^{2}$ & 0.001 & $-$0.007 & 0.020 & $-$0.010 & 0.0004 & 0.022 & 0.014 & 0.003 & 0.005 & 0.006 \\ 
Residual Std. Error (df = 98) & 0.455 & 0.467 & 0.498 & 0.567 & 0.507 & 0.520 & 0.496 & 0.488 & 0.494 & 0.463 \\ 
F Statistic (df = 1; 98) & 1.065 & 0.295 & 3.041$^{*}$ & 0.013 & 1.036 & 3.277$^{*}$ & 2.367 & 1.317 & 1.463 & 1.601 \\ 
\hline 
\hline \\[-1.8ex] 
P-valor & \multicolumn{10}{r}{$^{*}$p$<$0.1; $^{**}$p$<$0.05; $^{***}$p$<$0.01} \\ 
\end{tabular} 
\end{table} 

\end{landscape}

\newpage

\hypertarget{ejercicio-2.h}{%
\subsubsection{Ejercicio 2.h}\label{ejercicio-2.h}}

\textbf{Disminuya la varianza del ingreso permanente, y aumente la
varianza del ingreso transitorio y vuelva a estimar y graficar la
relación entre el consumo y el ingreso.}

Hasta ahora, los valores de las varianzas del ingreso permanente y el
transitorio eran, respectivamente: \(Var(Y^P) = Var(Y^T) = 4\). Ahora,
se utilizaron los valores \(Var'(Y^P) = .25\) y \(Var'(Y^T) = 9\). Una
vez realizado el procedimiento de código correspondiente, se encontró la
siguiente relación entre el Nuevo Consumo y el Nuevo Ingreso:

\includegraphics{Tarea_1_Macroeconomía-V2_files/figure-latex/REGRESIÓN7-1.pdf}

En lasa tablas que se presentan a continuación, se identifican los
siguientes comportamientos:

\begin{enumerate}
\def\labelenumi{\arabic{enumi}.}
\item
  El parámetro estimado \(\hat{\alpha}\) para este tercer modelo
  presenta una variación visiblemente menor a los dos anteriores. Esto
  significa que el ingreso permanente es bastante homogéneo entre
  individuos y no suele variar demasiado.
\item
  Por otra parte, se observa que la concentración de observaciones es
  menos compacta que en el caso anterior. Esto se debe a que la varianza
  del Ingreso Transitorio es más alta que antes, lo que implica una
  mayor dispersión de los datos alrededor del modelo estimado. Esto es,
  las relaciones entre ingreso y patrones de consumo no son tan
  consistentes y, de hecho, pueden encontrarse muchos más niveles
  diferentes de consumo para un solo nivel de ingreso que en los incisos
  anteriores.
\end{enumerate}

\newpage
\begin{landscape}


\begin{table}[!htbp] \centering 
  \caption{Tabla de Regresión 3} 
  \label{} 
\footnotesize 
\begin{tabular}{@{\extracolsep{5pt}}lcccccccccc} 
\\[-1.8ex]\hline 
\hline \\[-1.8ex] 
 & \multicolumn{10}{c}{Variable Dependiente} \\ 
\cline{2-11} 
\\[-1.8ex] & \multicolumn{10}{c}{Consumo} \\ 
 & 1 & 2 & 3 & 4 & 5 & 6 & 7 & 8 & 9 & 10 \\ 
\\[-1.8ex] & (1) & (2) & (3) & (4) & (5) & (6) & (7) & (8) & (9) & (10)\\ 
\hline \\[-1.8ex] 
 Ingreso & 0.108$^{***}$ & 0.108$^{***}$ & 0.072$^{***}$ & 0.077$^{***}$ & 0.115$^{***}$ & 0.096$^{***}$ & 0.054$^{***}$ & 0.135$^{***}$ & 0.088$^{***}$ & 0.095$^{***}$ \\ 
  & (0.020) & (0.015) & (0.015) & (0.014) & (0.019) & (0.017) & (0.017) & (0.017) & (0.017) & (0.018) \\ 
  & & & & & & & & & & \\ 
 Constante & 9.079$^{***}$ & 7.812$^{***}$ & 8.975$^{***}$ & 9.984$^{***}$ & 8.533$^{***}$ & 8.723$^{***}$ & 8.814$^{***}$ & 8.633$^{***}$ & 9.226$^{***}$ & 9.606$^{***}$ \\ 
  & (0.196) & (0.149) & (0.149) & (0.160) & (0.191) & (0.179) & (0.156) & (0.189) & (0.179) & (0.210) \\ 
  & & & & & & & & & & \\ 
\hline \\[-1.8ex] 
Observations & 100 & 100 & 100 & 100 & 100 & 100 & 100 & 100 & 100 & 100 \\ 
R$^{2}$ & 0.232 & 0.335 & 0.195 & 0.229 & 0.266 & 0.245 & 0.093 & 0.390 & 0.222 & 0.216 \\ 
Adjusted R$^{2}$ & 0.225 & 0.328 & 0.186 & 0.221 & 0.258 & 0.237 & 0.084 & 0.384 & 0.215 & 0.208 \\ 
Residual Std. Error (df = 98) & 0.555 & 0.489 & 0.450 & 0.472 & 0.509 & 0.478 & 0.463 & 0.527 & 0.485 & 0.498 \\ 
F Statistic (df = 1; 98) & 29.662$^{***}$ & 49.393$^{***}$ & 23.676$^{***}$ & 29.078$^{***}$ & 35.437$^{***}$ & 31.762$^{***}$ & 10.076$^{***}$ & 62.705$^{***}$ & 28.044$^{***}$ & 27.032$^{***}$ \\ 
\hline 
\hline \\[-1.8ex] 
P-valor & \multicolumn{10}{r}{$^{*}$p$<$0.1; $^{**}$p$<$0.05; $^{***}$p$<$0.01} \\ 
\end{tabular} 
\end{table} 

\begin{table}[!htbp] \centering 
  \caption{Tabla de Regresión 3 (Continuación)} 
  \label{} 
\footnotesize 
\begin{tabular}{@{\extracolsep{5pt}}lcccccccccc} 
\\[-1.8ex]\hline 
\hline \\[-1.8ex] 
 & \multicolumn{10}{c}{Variable Dependiente} \\ 
\cline{2-11} 
\\[-1.8ex] & \multicolumn{10}{c}{Consumo} \\ 
 & 11 & 12 & 13 & 14 & 15 & 16 & 17 & 18 & 19 & 20 \\ 
\\[-1.8ex] & (1) & (2) & (3) & (4) & (5) & (6) & (7) & (8) & (9) & (10)\\ 
\hline \\[-1.8ex] 
 Ingreso & 0.116$^{***}$ & 0.120$^{***}$ & 0.104$^{***}$ & 0.106$^{***}$ & 0.071$^{***}$ & 0.111$^{***}$ & 0.127$^{***}$ & 0.073$^{***}$ & 0.087$^{***}$ & 0.118$^{***}$ \\ 
  & (0.017) & (0.016) & (0.016) & (0.019) & (0.018) & (0.019) & (0.017) & (0.016) & (0.017) & (0.017) \\ 
  & & & & & & & & & & \\ 
 Constante & 9.936$^{***}$ & 9.772$^{***}$ & 8.824$^{***}$ & 9.574$^{***}$ & 8.619$^{***}$ & 8.626$^{***}$ & 8.035$^{***}$ & 9.190$^{***}$ & 9.135$^{***}$ & 9.044$^{***}$ \\ 
  & (0.196) & (0.185) & (0.163) & (0.213) & (0.173) & (0.182) & (0.172) & (0.159) & (0.174) & (0.184) \\ 
  & & & & & & & & & & \\ 
\hline \\[-1.8ex] 
Observations & 100 & 100 & 100 & 100 & 100 & 100 & 100 & 100 & 100 & 100 \\ 
R$^{2}$ & 0.327 & 0.359 & 0.303 & 0.237 & 0.142 & 0.261 & 0.357 & 0.178 & 0.217 & 0.318 \\ 
Adjusted R$^{2}$ & 0.320 & 0.352 & 0.296 & 0.229 & 0.133 & 0.254 & 0.351 & 0.170 & 0.209 & 0.311 \\ 
Residual Std. Error (df = 98) & 0.463 & 0.464 & 0.499 & 0.570 & 0.510 & 0.522 & 0.491 & 0.480 & 0.493 & 0.461 \\ 
F Statistic (df = 1; 98) & 47.608$^{***}$ & 54.771$^{***}$ & 42.606$^{***}$ & 30.359$^{***}$ & 16.185$^{***}$ & 34.623$^{***}$ & 54.463$^{***}$ & 21.276$^{***}$ & 27.226$^{***}$ & 45.641$^{***}$ \\ 
\hline 
\hline \\[-1.8ex] 
P-valor & \multicolumn{10}{r}{$^{*}$p$<$0.1; $^{**}$p$<$0.05; $^{***}$p$<$0.01} \\ 
\end{tabular} 
\end{table} 

\end{landscape}
\newpage

\hypertarget{ejercicio-3}{%
\subsection{Ejercicio 3}\label{ejercicio-3}}

\textbf{Estudie el consumo agregado en México siguiendo estos pasos:
{[}3 horas, 0.5 puntos cada inciso{]}}

\hypertarget{ejercicio-3.a}{%
\subsection{Ejercicio 3.a}\label{ejercicio-3.a}}

\textbf{Obtenga, del Inegi, datos de ``C'', el consumo agregado en
México, de ``Y'', el producto agregado, de ``I'', la inversión agregada,
de ``G'', el gasto del gobierno y de , de ``NX'', las exportaciones
netas, entre 1980 y el tercer trimestre de 2019, EN TÉRMINOS REALES.}

Para este ejercicio obtuvimos datos del INEGI de 1980 a 1995 a precios
de 1980, y de 1993 a 2019 a precios de 2013, para lo cual deflactamos
los datos que se encontraban a precios de 1980 para tener todos los
datos a precios de 2013 expresados en miles de millones de pesos.

\includegraphics{Tarea_1_Macroeconomía-V2_files/figure-latex/unnamed-chunk-3-1.pdf}

\newpage

\hypertarget{ejercicio-3.b}{%
\subsubsection{Ejercicio 3.b}\label{ejercicio-3.b}}

\textbf{Grafíque dichas serie de tiempo juntas para compararlas
visualmente. (Compare la gráfica de las variables (de las que son
siempre positivas) en su valor real original, y después de sacarles el
logaritmo (cualquier logaritmo, no hace diferencia\ldots)).}

\includegraphics{Tarea_1_Macroeconomía-V2_files/figure-latex/unnamed-chunk-6-1.pdf}

\includegraphics{Tarea_1_Macroeconomía-V2_files/figure-latex/unnamed-chunk-8-1.pdf}

En las gráficas anteriores observamos la trayectoria de las variables
macroeconómicas de la economía mexicana. La trayectoria del ingreso
medido por el PIB en el periodo 1980 - 2019 representa al ingreso
permanente puesto que está en el largo plazo. Es observable como el
consumo agregado sigue una trayectoria muy cercana a la del ingreso
permanente y es más notorio al suavizar la regresión aplicando
logaritmos.

\newpage

\hypertarget{ejercicio-3.c}{%
\subsubsection{Ejercicio 3.c}\label{ejercicio-3.c}}

\textbf{Grafique también la tasa de crecimiento,
\(\% \Delta Y_t =(a_t-a_{t-1})/a_{t-1}\) , de todas las series}

Para este ejercicio se calcularon tasas de crecimiento de las variables
respecto del mismo trimestre de un año anterior. Los resultados son los
siguientes:

\includegraphics{Tarea_1_Macroeconomía-V2_files/figure-latex/unnamed-chunk-9-1.pdf}

\includegraphics{Tarea_1_Macroeconomía-V2_files/figure-latex/unnamed-chunk-10-1.pdf}

La tasa de crecimiento del consumo agregado a oscilado alrededor de cero
en este periodo, pero con fuertes caídas en las fechas en que hubo
crisis económicas.

Podemos observar que la tasa de crecimiento del consumo y del ingreso
son muy parecidas, mantienen una similitud muy fuerte, yendo de la mano
de la hipótesis de la renta permanente.

\includegraphics{Tarea_1_Macroeconomía-V2_files/figure-latex/unnamed-chunk-11-1.pdf}

La tasa del crecimiento del Gasto público también tiene un
comportamiento oscilatorio, sin embargo, en 1992 tuvo una fuerte caída
por metas para sanear las finanzas públicas y como consecuencia de un
menor servicio de la deuda logrando un balance financiero de caja
superavitario, incluyendo ingresos extraordinarios (por la privatización
de empresas) (Informe anual Banco de México 1992). Para 1993 se aumentó
el gasto público ``dedicado a promover el desarrollo social y a apoyar a
las clases menos favorecidas del país'' (Informe anual Banco de México
1993.

~

\includegraphics{Tarea_1_Macroeconomía-V2_files/figure-latex/unnamed-chunk-12-1.pdf}

\includegraphics{Tarea_1_Macroeconomía-V2_files/figure-latex/unnamed-chunk-13-1.pdf}

Por lo regular las exportaciones netas para México son negativas, pero
la caída es más fuerte cuando han habido crisis en la economía mundial.

\hypertarget{ejercicio-3.d}{%
\subsubsection{Ejercicio 3.d}\label{ejercicio-3.d}}

\textbf{Enfóquese ahora nada más a los cambios porcentuales en el
consumo y el producto agregados: grafique la relación entre una serie y
otra, es decir, grafique los puntos \((\% \Delta Y_t , \% \Delta C_t)\)
poniendo el consumo en las ordenadas.}

\includegraphics{Tarea_1_Macroeconomía-V2_files/figure-latex/unnamed-chunk-14-1.pdf}

Hay una relación directa entre la tasa de crecimiento del consumo y y la
tasa de crecimiento del ingreso, por la gráfica sabemos que la
elasticidad entre ambas variables es grande, esto es, que un cambio
porcentual de 1\% del ingreso permanente ocasiona un cambio porcentual
del consumo -no igual a 1\%- pero si cercano a 1\%.

\hypertarget{ejercicio-3.e}{%
\subsubsection{Ejercicio 3.e}\label{ejercicio-3.e}}

\textbf{Calcule la volatilidad de ambas series de tasas de crecimiento.}

Para calcular la volatilidad de las tasas de crecimiento del PIB y
consumo, se utilizó el indicador de desviación estándar, que mide el
grado de dispersión de las observaciones respecto a su media, de acuerdo
a la siguiente fórmula:

\[\sigma=\sqrt{\frac {1}{N}\sum_{i=1}^{N}(x_i-\mu)^2}\] Obteniendo los
siguientes resultados:

\begin{center}
\textbf{Cuadro 3.2. Desviaciones estándar del PIB y el Consumo}
\end{center}

\begin{longtable}[]{@{}ccc@{}}
\toprule
Indicador & PIB & Consumo \\
\midrule
\endhead
Desviación Estándar & 3.4716 & 3.8565 \\
\bottomrule
\end{longtable}

Con el cuadro anterior, podemos observar que la desviación estándar de
la tasa de crecimiento de PIB es de 3.47 puntos porcentuales y de la
tasa de crecimiento de consumo, es de 3.86 puntos porcentuales, por lo
que la desviación estándar de la tasa de crecimiento de consumo es mayor
que la desviación estándar de la tasa de crecimiento del PIB.

Asimismo, se procedió a realizar el gráfico que mida la variación de los
datos respecto a su media, encontrando que ambas series presentan
volatilidad en el periodo de estudio.

~

\includegraphics{Tarea_1_Macroeconomía-V2_files/figure-latex/unnamed-chunk-16-1.pdf}
\newpage

\hypertarget{ejercicio-3.f}{%
\subsubsection{Ejercicio 3.f}\label{ejercicio-3.f}}

\textbf{Estime cuatro modelos lineales: \(C_t=a+bY_t+\epsilon_t\),
\(\Delta \%C_t=a+b\Delta \%Y_{t}+\epsilon_t\),
\(\Delta \%C_t=a+b\Delta \%Y_{t-1}+\epsilon_t\) y
\(c_t=a+by+\epsilon_t\), donde las minúsculas reflejan el logaritmo de
la variable en mayúscula, y reporte los valores estimados de los
coeficientes, los estadísticos T, las R cuadradas, etc. Para resolver el
ejercicio, procedimos a resolver las cuatro regresiones.}

\begin{center}
\textbf{Modelo 1. $C_t=a+bY_t+\epsilon_t$}
\end{center}

En esta regresión, utilizamos como variable independiente, el PIB (Y) y
variable dependiente, el consumo (C), observando una relación positiva
entre PIB y consumo, obteniendo el siguiente resultado.

\[\hat C_t=-681.85+0.71 Y_t+\epsilon_t \] Esta regresión nos muestra que
por cada 100 pesos de PIB (ingreso), el individuo destina 71 pesos al
consumo. Respecto a los estadísticos obtenidos, el p-valor es 0.007,
menor a 0.05, donde se rechaza la hipótesis nula, concluyendo que el
ingreso es estadísticamente significativo para explicar el consumo. Con
relación al \(R^2\), este indicador es 0.987, es decir que el PIB
explica en un 98.7 por ciento el comportamiento del consumo (ver Cuadro
2).

\includegraphics{Tarea_1_Macroeconomía-V2_files/figure-latex/unnamed-chunk-17-1.pdf}

\begin{center}
\textbf{Modelo 2. $\Delta \% \hat C_t=a+b\Delta \%Y_{t}+\epsilon_t$}
\end{center}

Para realizar la regresión, se considero como variable independiente, la
tasa de crecimiento del PIB (\(\Delta \%Y\)) y variable dependiente, la
tasa de crecimiento del consumo (\(\Delta \%C\)).

Las variables muestran una relación positiva, es decir, si la tasa de
crecimiento del PIB se incrementa en 10 puntos porcentuales , la tasa de
crecimiento del consumo crece en un 9.48 puntos porcentuales, tal como
se muestra los resultados en la siguiente ecuación:

\[\Delta \% \hat C_t=0.21+0.948\Delta \%Y_{t}+\epsilon_t\]

Con relación, a los estadísticos obtenidos, el p-valor es 0.047, menor a
0.05, es decir la tasa de crecimiento del PIB es estadísticamente
significativa para explicar el comportamiento del consumo. El \(R^2\) es
0.729, indicando que el PIB explica en un 72.9 por ciento la evolución
del consumo (ver Cuadro 2).

\includegraphics{Tarea_1_Macroeconomía-V2_files/figure-latex/unnamed-chunk-18-1.pdf}

\begin{center}
\textbf{Modelo 3. $\Delta \%C_t=a+b\Delta \%Y_{t-1}+\epsilon_t$}
\end{center}

La regresión considera como variable independiente la tasa de
crecimiento del PBI rezagado por un periodo y la variable dependiente es
la tasa de crecimiento del consumo.

Las variables utilizadas muestran una relación positiva, la cual
presentamos con la siguiente regresión:

\[\Delta \% \hat C_t=0.5782+0.7711\Delta \%Y_{t-1}+\epsilon_t\]

\includegraphics{Tarea_1_Macroeconomía-V2_files/figure-latex/unnamed-chunk-20-1.pdf}

\newpage

~

~

\begin{center}
\textbf{Modelo 4. $c_t=a+by_t+\epsilon_t$}
\end{center}

Recordar que las minúsculas reflejan el logaritmo de la variable en
mayúscula. Para realizar la regresión, se considero como variable
independiente, el logaritmo del PIB (\(\Delta \%Y\)) y variable
dependiente, el logaritmo del consumo (\(\Delta \%C\)).

~

Las variables muestran una relación positiva, como se muestra los
resultados en la siguiente ecuación:

\[\hat c_t=-1.285+1.091y_t+\epsilon_t\]

~

Con relación a los estadísticos, el p-valor es 0.011, menor a 0.05,
rechazando la hipótesis nula, es decir que el logaritmo del PIB es
estadísticamente significativa para explicar el logaritmo del consumo.
Respecto al \(R^2\), el logaritmo del PIB explica en un 98.5 por ciento
el logaritmo del consumo (ver Cuadro 1).

~

\includegraphics{Tarea_1_Macroeconomía-V2_files/figure-latex/unnamed-chunk-21-1.pdf}

\begin{table}[!htbp] \centering 
  \caption{Tabla de Regresión 4. Consumo e Ingreso.} 
  \label{} 
\scriptsize 
\begin{tabular}{@{\extracolsep{5pt}}lcccc} 
\\[-1.8ex]\hline 
\hline \\[-1.8ex] 
 & \multicolumn{4}{c}{Variable Dependiente} \\ 
\cline{2-5} 
\\[-1.8ex] & C & \multicolumn{2}{c}{tcC} & crecC, logC \\ 
\\[-1.8ex] & \textit{OLS} & \textit{OLS} & \textit{dynamic} & \textit{OLS} \\ 
 & \textit{} & \textit{} & \textit{linear} & \textit{} \\ 
\\[-1.8ex] & (1) & (2) & (3) & (4)\\ 
\hline \\[-1.8ex] 
 Y & 0.713$^{***}$ &  &  &  \\ 
  & (0.007) &  &  &  \\ 
  & & & & \\ 
 tcY &  & 0.948$^{***}$ &  &  \\ 
  &  & (0.047) &  &  \\ 
  & & & & \\ 
 crecY &  &  & 0.771$^{***}$ &  \\ 
  &  &  & (0.064) &  \\ 
  & & & & \\ 
 logY &  &  &  & 1.091$^{***}$ \\ 
  &  &  &  & (0.011) \\ 
  & & & & \\ 
 Constant & $-$681.855$^{***}$ & 0.210 & 0.578$^{**}$ & $-$1.285$^{***}$ \\ 
  & (83.516) & (0.196) & (0.269) & (0.102) \\ 
  & & & & \\ 
\hline \\[-1.8ex] 
Observations & 159 & 155 & 154 & 159 \\ 
R$^{2}$ & 0.987 & 0.729 & 0.489 & 0.985 \\ 
Adjusted R$^{2}$ & 0.987 & 0.727 & 0.485 & 0.985 \\ 
Residual Std. Error & 278.509 (df = 157) & 2.014 (df = 153) & 2.752 (df = 152) & 0.038 (df = 157) \\ 
F Statistic & 11,964.010$^{***}$ (df = 1; 157) & 411.461$^{***}$ (df = 1; 153) & 145.273$^{***}$ (df = 1; 152) & 10,136.240$^{***}$ (df = 1; 157) \\ 
\hline 
\hline \\[-1.8ex] 
P-valor & \multicolumn{4}{r}{$^{*}$p$<$0.1; $^{**}$p$<$0.05; $^{***}$p$<$0.01} \\ 
\end{tabular} 
\end{table}

\hypertarget{ejercicio-3.g}{%
\subsubsection{Ejercicio 3.g}\label{ejercicio-3.g}}

\textbf{Explique qué se puede concluir a cerca de la Hipótesis de
Ingreso Permanente para México a partir de los coeficientes
encontrados.}

~

La hipótesis de ingreso permanente plantea que el individuo tiene un
consumo homogéneo a lo largo de su vida, por eso los individuos no
consumen respecto a sus ingresos transitorios sino respecto a las
expectativas de sus ingresos.

~

Con relación a los resultados encontrados, podemos concluir que si
cumple la hipótesis del ingreso permanente en México, debido a que los
cuatro modelos de regresión muestran una relación positiva entre ingreso
y consumo con p-valor y \(R^2\) estadísticamente significativo.

\newpage

\hypertarget{ejercicio-4}{%
\subsection{Ejercicio 4}\label{ejercicio-4}}

\hypertarget{ejercicio-4.a}{%
\subsubsection{Ejercicio 4.a}\label{ejercicio-4.a}}

\textbf{Baje los datos de un año de la ENIGH del sitio del INEGI, (Grupo
1-2018, Grupo 2-2016, etc.) y establezca el número de hogares y el
ingreso y el gasto promedio.}

Al inspeccionar las bases de microdatos de ``Concentrado de hogares'' de
la Encuesta Nacional de Ingresos y Gastos de los Hogares 2016, se
observó que el número de hogares (cada hogar está identificado por un
folio) es de 70311. A continuación se muestran la media de ambas
variables:

\begin{center}
\textbf{Cuadro 4.1. Obervaciones y media de la ENIGH 2016}
\end{center}

\begin{longtable}[]{@{}lll@{}}
\toprule
& Gastos trimestrales & Ingresos trimestrales \\
\midrule
\endhead
Número de Observaciones (Hogares) & 70311 & 70311 \\
Media (en MXN) & 25589.98 & 42038.99 \\
\bottomrule
\end{longtable}

\hypertarget{ejercicio-4.b}{%
\subsubsection{Ejercicio 4.b}\label{ejercicio-4.b}}

\textbf{Estime una relación entre ingreso y gasto y reporte sus
resultados.}

Para realizar esta observación, se eliminaron datos considerados
\emph{outliers} tanto para ingreso como para gasto. Específicamente, no
se tomaron en cuenta ingresos superiores al millón de pesos trimestral y
a los 500'000 pesos trimestrales, respectivamente; así, el número de
observaciones utilizadas para este inciso es de 70'280. Lo anterior con
el objetivo de que el modelo lineal estimado se comportara mejor:

\includegraphics{Tarea_1_Macroeconomía-V2_files/figure-latex/REGRESIÓN GASTOS INGRESOS-1.pdf}

\begin{table}[!htbp] \centering 
  \caption{Tabla de Regresión 5. Ingreso y Gasto trimestral, 2016} 
  \label{} 
\footnotesize 
\begin{tabular}{@{\extracolsep{5pt}}lc} 
\\[-1.8ex]\hline 
\hline \\[-1.8ex] 
 & \multicolumn{1}{c}{Variable Dependiente} \\ 
\cline{2-2} 
\\[-1.8ex] & Gasto \\ 
\hline \\[-1.8ex] 
 Ingreso & 1.190$^{***}$ \\ 
  & (0.005) \\ 
  & \\ 
 Constante & 10.374$^{***}$ \\ 
  & (0.163) \\ 
  & \\ 
\hline \\[-1.8ex] 
Observations & 70,280 \\ 
R$^{2}$ & 0.480 \\ 
Adjusted R$^{2}$ & 0.480 \\ 
Residual Std. Error & 29.584 (df = 70278) \\ 
F Statistic & 64,761.990$^{***}$ (df = 1; 70278) \\ 
\hline 
\hline \\[-1.8ex] 
P-valor & \multicolumn{1}{r}{$^{*}$p$<$0.1; $^{**}$p$<$0.05; $^{***}$p$<$0.01} \\ 
\end{tabular} 
\end{table}

\newpage

\hypertarget{ejercicio-4.c}{%
\subsubsection{Ejercicio 4.c}\label{ejercicio-4.c}}

\textbf{Estime una relación entre ingreso y gasto pero para hogares
unipersonales de edad entre 40 y 50 años de edad de la Ciudad de
México.}

Al obtener las observaciones correspondientes a los hogares
unipersonales de 40 a 50 años en la Ciudad de México de la base de datos
concentrada de la ENIGH 2016, surgió el problema de solo existir dos
observaciones bajo estos filtros. Por tal motivo, y con el objetivo de
presentar una estimación sustantiva de la relación entre el ingreso y el
gasto de un grupo particular de población, a continuación se exhibe un
modelo lineal estimado para los \textbf{hogares unipersonales y
nucleares en la Ciudad de México, de entre 40 y 50 años} (donde un hogar
nuclear está compuesto por una pareja, pareja con hijos ó persona con
hijos).

De igual manera que el inciso anterior, se eliminaron las observaciones
\emph{outliers}: en este caso, solo se trató de una, por lo que la
regresión que se presenta a continuación estima un total de 70 hogares:

\includegraphics{Tarea_1_Macroeconomía-V2_files/figure-latex/REGRESIÓN HOGARESUNI-1.pdf}

\begin{table}[!htbp] \centering 
  \caption{Tabla de Regresión 6. Ingreso y Gasto trimestral, 2016} 
  \label{} 
\footnotesize 
\begin{tabular}{@{\extracolsep{5pt}}lc} 
\\[-1.8ex]\hline 
\hline \\[-1.8ex] 
 & \multicolumn{1}{c}{Variable Dependiente} \\ 
\cline{2-2} 
\\[-1.8ex] & Gasto \\ 
\hline \\[-1.8ex] 
 Ingreso & 1.018$^{***}$ \\ 
  & (0.118) \\ 
  & \\ 
 Constante & 2.416 \\ 
  & (1.997) \\ 
  & \\ 
\hline \\[-1.8ex] 
Observations & 70 \\ 
R$^{2}$ & 0.524 \\ 
Adjusted R$^{2}$ & 0.517 \\ 
Residual Std. Error & 9.681 (df = 68) \\ 
F Statistic & 74.860$^{***}$ (df = 1; 68) \\ 
\hline 
\hline \\[-1.8ex] 
P-valor & \multicolumn{1}{r}{$^{*}$p$<$0.1; $^{**}$p$<$0.05; $^{***}$p$<$0.01} \\ 
\end{tabular} 
\end{table}

\hypertarget{ejercicio-4.d}{%
\subsubsection{Ejercicio 4.d}\label{ejercicio-4.d}}

\textbf{Interprete sus resultados.}

De acuerdo a los resultados de los dos incisos anteriores puede
observarse que la responsividad del gasto ante cambios en el ingreso en
Hogares Unipersonales y Nucleares de 40 a 50 años en la Ciudad de México
es \textbf{menor} a aquella de los hogares nacionales.

Intuitivamente, esto puede deberse a que los hogares unipersonales y
nucleares de 40 a 50 años en la Ciudad de México, se encuentran en un
contexto de su vida donde sus gastos son menos volátiles y se encuentran
relativamente bien definidos de acuerdo a sus preferencias o los
integrantes del hogar. Por otra parte, la muestra nacional también
incluye hogares donde una variación del ingreso, dependiendo de la edad
y el contexto familiar, podría verse reflejado en una modificación más
sustancial de la estructura de gastos del hogar.

\hypertarget{ejercicio-4.e}{%
\subsubsection{Ejercicio 4.e}\label{ejercicio-4.e}}

\textbf{Para todos los hogares unipersonales, estime el valor promedio
del ingreso por edad, separando la muestra en grupos de edad de cinco
años cada uno y grafiquelo.}

Utilizando la base de datos completa de la ENIGH 2016, se filtraron las
observaciones pertinentes para este ejercicio, a saber: el tipo de
hogar. Una vez hecho esto, se hizo uso de Excel para obtener los
promedios de ingreso y gasto por hogar en grupos de edad de 5 años.

Posteriormente se importó la base de datos a R y se obtuvieron las
gráficas que a continuación se presentan:

\includegraphics{Tarea_1_Macroeconomía-V2_files/figure-latex/unnamed-chunk-23-1.pdf}

\includegraphics{Tarea_1_Macroeconomía-V2_files/figure-latex/unnamed-chunk-24-1.pdf}

En estas gráficas pueden analizarse algunos elementos interesantes:

\begin{enumerate}
\def\labelenumi{\arabic{enumi}.}
\item
  Es claro que el ingreso promedio trimestral para hogares unipersonales
  tiene sus puntos más altos de los 26 hasta los 45 años. Esto debido a
  que se trata de individuos adultos jóvenes, potencialmente muy
  productivos, que viven solos, lo cual incrementa sustancialmente el
  ingreso promedio del hogar.
\item
  Por otro lado, a partir de los 71 años, se observa que los ingresos
  promedios trimestrales de hogares unilaterales caen por debajo de la
  media. Si suponemos que se trata de adultos mayores cuya productividad
  y oportunidades laborales son más bien escasas, el comportamiento
  tiene sentido.
\item
  En el caso del gasto promedio trimestral de hogares unipersonales, la
  tendencia es, si cabe, aún mucho más evidente: Los jóvenes adultos en
  etapa productiva tienen gastos sustancialmente altos. De hecho, se
  evidencia una tendencia consistentemente decreciente en el gasto
  trimestral cada lustro.
\end{enumerate}

\hypertarget{ejercicio-4.f}{%
\subsubsection{Ejercicio 4.f}\label{ejercicio-4.f}}

\textbf{Explique qué esperaría ver con los datos de 2020 acerca de la
relación entre consumo e ingreso para los hogares mexicanos.}

De acuerdo a los comportamientos de las variables estudiadios en este
ejercicio y los anteriores, la fuerte crisis que atraviesa México se ha
reflejado en un decremento significativo del ingreso nacional; a pesar
de que se ha observado cierta recuperación en los últimos trimestres, lo
cierto es que el impacto en la renta del país fue, y sigue siendo, de
los más severos de la historia económica reciente.

Por su parte, las relaciones ya establecidas a lo largo de este
documento indican una clara y consistente relación entre los niveles de
ingreso, gasto y consumo agregados, pero también en grupos demográficos
específicos. Aunque, dependiendo del caso, la responsividad o estrechez
de la relación puede variar, se ha evidenciado que la relación es
innegablemente positiva. Por tal motivo, las recientes coyunturas
económicas de México, con toda probabilidad, provocarán caídas
sustanciales en los niveles de consumo agregado y en el gasto trimestral
que realiza cualquier tipo de hogar.

\newpage

\hypertarget{ejercicio-5}{%
\subsection{Ejercicio 5}\label{ejercicio-5}}

\textbf{Estudie el acertijo del premio al riesgo para el caso de Mexico
siguiendo los pasos a - h. {[}3 horas, 0.5 puntos cada inciso{]}}

Una de las implicaciones más importantes del análisis de los
rendimientos esperados de los activos se refiere al caso en el que el
activo de riesgo es una amplia cartera de acciones. Si asumimos que los
individuos tienen una utilidad de aversión al riesgo relativo constante,
la ecuación de Euler puede expresarse como
\(C_t^{-\theta}=\frac{1}{1+\rho}E_t[(1+r^i_{t+1})C^{-\theta}_{t+1}]\),
donde \(\theta\) es el \emph{coeficiente de aversión relativa al
riesgo}. Siguiendo a Romer\footnote{Romer, Advanced Macroeconomics,
  McGraw-Hill Education, 5 dition, p.397} tenemos que la diferencia
entre los retornos esperados de dos activos, \emph{i} y \emph{j},
satisface la ecuación \(E[r^i]-E[r^j]=\theta Cov(r^i-r^j, g^c)\), donde
\(g^c\) es la tasa de crecimiento del consumo de un periodo t al periodo
t+1, \((\frac{C_{t+1}}{C_t})-1\).

\hypertarget{ejercicio-5.a}{%
\subsubsection{Ejercicio 5.a}\label{ejercicio-5.a}}

\textbf{Consiga los valores anuales de IPC, el Indice de Precios y
Cotizaciones de la Bolsa Mexicana de Valores por lo menos desde 1990.}

En la siguiente gráfica se muestra la evolución de los valores de cierre
anuales del Índice de Precios y Cotizaciones de la Bolsa Mexicana de
Valores (S\&P/BMV IPC) desde 1990 hasta el año 2020, con año base de
1978. La tendencia de largo plazo es creciente, y en los últimos años
las caídas más drásticas son en los años 2008 y 2018. Se oberva también
un máximo histórico del índice en el año 2017, con un valor de
49,354.40.

\begin{verbatim}
## Warning: Missing column names filled in: 'X10' [10], 'X11' [11], 'X12' [12],
## 'X13' [13], 'X14' [14], 'X15' [15]
\end{verbatim}

\begin{verbatim}
## 
## -- Column specification --------------------------------------------------------
## cols(
##   year = col_double(),
##   IPCc = col_number(),
##   tcIPCc = col_double(),
##   CETES28 = col_double(),
##   CETES364 = col_double(),
##   Cons = col_double(),
##   tcCons = col_double(),
##   CM = col_double(),
##   tcCM = col_double(),
##   X10 = col_logical(),
##   X11 = col_logical(),
##   X12 = col_logical(),
##   X13 = col_logical(),
##   X14 = col_logical(),
##   X15 = col_logical()
## )
\end{verbatim}

\includegraphics{Tarea_1_Macroeconomía-V2_files/figure-latex/unnamed-chunk-25-1.pdf}

\newpage

\hypertarget{ejercicio-5.b}{%
\subsubsection{Ejercicio 5.b}\label{ejercicio-5.b}}

\textbf{Calcule su tasa de retorno nominal para cada año.}

La tasa de retorno de un activo i es
\(r^i_{t+1}=\frac{D^i_{t+1}}{P^i_t}-1\), donde \(D^i_{t+1}\) es el pago
del activo i, que incluye pagos de dividendos y beneficios por la venta
del activo en t+1, y \(P^i_t\) es el precio inicial del activo.

La siguiente gráfica muestra la evolución de la tasa de retorno del IPC,
de 1991 a 2020. Para el cálculo se tomó la tasa de crecimiento del
índice de un año a otro. Se observa una mayor variabilidad en los
primeros años considerados, y más estabilidad en los últimos años. Las
caídas más fuertes, coinciden con los acontecimientos que más han
impactado de manera negativa en la economía mexicana, y que han tenido
su orígen en el mercado financiero doméstico e internacional, como por
ejemplo en 1994, 1998, 2008 y 2018.

\includegraphics{Tarea_1_Macroeconomía-V2_files/figure-latex/unnamed-chunk-26-1.pdf}

\newpage

\hypertarget{ejercicio-5.c}{%
\subsubsection{Ejercicio 5.c}\label{ejercicio-5.c}}

\textbf{Consiga los valores promedio anual de la tasa de interés de
CETES a 28 días, o la TIIE, la tasa interbancaria de equilibrio, y de la
tasa de interés a un año, para el periodo que esté disponible.}

Para el corto plazo se tomó la tasa de interés de CETES a 28 días y para
el largo plazo se tomó la tasa de CETES a 364 días. En la siguiente
gráfica se muestra la evolución de ambas tasas, éstas muestran una
tendencia descendente entre 1990 y 2020. Se observa un incremento
drástico en 1995 con una tasa de 48.65\% debido a la crisis financiera
originada internamente a finales de 1994 y principios de 1995,

\includegraphics{Tarea_1_Macroeconomía-V2_files/figure-latex/unnamed-chunk-27-1.pdf}

\newpage

\hypertarget{ejercicio-5.d}{%
\subsubsection{Ejercicio 5.d}\label{ejercicio-5.d}}

\textbf{Calcule la diferencia entre el retorno del IPC y el retorno de
invertir en CETES a distintos plazos.}

La siguiente gráfica muestra el diferencial de los retornos del IPC y de
invertir en CETES a 28 días y el diferencial de los retornos del IPC y
de invertir en CETES A 364 días. Se observa que el diferencial (exceso
de rendimiento) para ambos casos es más estable en la última década. Se
observan caídas importantes del exceso de rendimiento con respecto a los
activos sin riesgo en 1995, 1998, 2000, 2008, y 2018.

\includegraphics{Tarea_1_Macroeconomía-V2_files/figure-latex/unnamed-chunk-28-1.pdf}

\newpage

\hypertarget{ejercicio-5.e}{%
\subsubsection{Ejercicio 5.e}\label{ejercicio-5.e}}

\textbf{Calcule la covarianza entre dicha diferencias y la tasa de
crecimiento real del consumo agregado de la economía mexicana.}

En la siguiente gráfica se muestra la evolución de la tasa de
crecimiento del consumo (\(g^c\)) de la economía mexicana de 1991 a
2020. Las caídas más drásticas se obervan en 1995, 2009 y 2020, siendo
sta última la caida más grande del periodo, con una tasa de -11.66\%,
casi del doble de la caída de 1995 y 2009.

\includegraphics{Tarea_1_Macroeconomía-V2_files/figure-latex/unnamed-chunk-29-1.pdf}

Para calcular la covarianza de la tasa de crecimiento del consumo
(\(g^c\)), y del diferencial del retorno del IPC con el retorno de los
CETES, planteamos dos cálculos, uno para las tasas de CETES a 28 días y
otro para las tasas de CETES a 364 días.

\begin{enumerate}
\def\labelenumi{\roman{enumi})}
\tightlist
\item
  \(cov(r^i-r_{cp}^j, g^c)\), donde \(r^i\) es la tasa de retorno del
  IPC y \(r_{cp}^j\) la tasa libre de riesgos de los CETES a 28 días.
\end{enumerate}

El cálculo se obtiene como
\(cov(r^i-r_{cp}^j, g^c)=corr(r^i-r_{cp}^j, g^c)*sd(g^c)*sd(r^i-r_{cp}^j)\).
Donde \(corr(r^i-r_{cp}^j, g^c)\) es la correlación entre el diferencial
de retornos (exceso de rendimiento en el mercado de activos) y la tasa
de crecimiento del consumo, \(sd(g^c)\) la desviación estándar de la
tasa de crecimiento del consumo y \(sd(r^i-r_{cp}^j)\) es la desviación
estándar del diferencial de retornos.

\begin{center}
\textbf{Cuadro 5.1. Estadísticas descriptivas 28 días}
\end{center}

\begin{table}[h!]
                \begin{center} 
                    \begin{tabular}{|c|c|c|c|} 
                        \hline 
                        \mbox{}\;\;\;\;\;\mbox{}& \; Notación\;  & \; Valor calculado\; \\ 
                        \hline 
                        \multirow{4}{*}{\begin{sideways}28 días\end{sideways}}
                        &  &   \\
                        & $sd(g^c)$  & $0.04001254$  \\ 
                        &$sd(r^i-r_{cp}^j)$ & $0.3243786$ \\ 
                        &$corr(r^i-r_{cp}^j, g^c)$ & $0.07104924$ \\ 
                        &$cov(r^i-r_{cp}^j, g^c)$  & $0.0009221631$ \\                      
                        \hline \hline
                        \hline
                    \end{tabular}
                \end{center}
\end{table}

\begin{enumerate}
\def\labelenumi{\roman{enumi})}
\setcounter{enumi}{1}
\tightlist
\item
  \(cov(r^i-r_{lp}^j, g^c)\), donde \(r^i\) es la tasa de retorno del
  IPC y \(r_{lp}^j\) la tasa libre de riesgos de los CETES a 364 días.
\end{enumerate}

El cálculo se obtiene como
\(cov(r^i-r_{lp}^j, g^c)=corr(r^i-r_{lp}^j, g^c)*sd(g^c)*sd(r^i-r_{lp}^j)\).
Donde \(corr(r^i-r_{lp}^j, g^c)\) es la correlación entre el diferencial
de retornos (exceso de rendimiento en el mercado de activos) y la tasa
de crecimiento del consumo, \(sd(g^c)\) la desviación estándar de la
tasa de crecimiento del consumo y \(sd(r^i-r_{lp}^j)\) es la desviación
estándar del diferencial de retornos.

\begin{center}
\textbf{Cuadro 5.2. Estadísticas descriptivas 364 días}
\end{center}

\begin{table}[h!]
                \begin{center} 
                    \begin{tabular}{|c|c|c|c|} 
                        \hline 
                        \mbox{}\;\;\;\;\;\mbox{}& \; Notación\;  & \; Valor calculado\; \\ 
                        \hline 
                        \multirow{4}{*}{\begin{sideways}364 días\end{sideways}}
                        &  &   \\
                        & $sd(g^c)$  & $0.04001254$  \\ 
                        &$sd(r^i-r_{lp}^j)$ & $0.317422$ \\ 
                        &$corr(r^i-r_{lp}^j, g^c)$ & $0.03890785$ \\ 
                        &$cov(r^i-r_{lp}^j, g^c)$  & $0.0004941631$ \\                      
                        \hline \hline
                        \hline
                    \end{tabular}
                \end{center}
\end{table}

Si a la \(j\) le corresponden los bonos del gobierno (CETES) y a \(i\)
le corresponden las inversiones en la bolsa (S\&P/IPC BMV), entonces la
covarianza del exceso de rendimientos \(r^i-r^j\) con \(g^c\), es de
\(0.0009221631\) con CETES a 28 días, y de \(0.0004941631\) con CETES a
364 días. Para el primer caso, la volatilidad del consumo es de \(4\)
por ciento y la volatilidad del diferencial de retornos de las acciones
con CETES a 28 días es \(32.43\) por ciento. Para es segundo caso, la
volatilidad del diferencial de retornos de las acciones con CETES a 364
días es \(31.74\) por ciento.

\newpage

\hypertarget{ejercicio-5.f}{%
\subsubsection{Ejercicio 5.f}\label{ejercicio-5.f}}

\textbf{Calcule el valor de aversión relativa al riesgo que implican
estos números, dado el supuesto de una utilidad con forma ARRC
(funciones de utilidad de la Aversión Relativa al Riesgo Constante).}

Recordemos que la diferencia entre los retornos del mercado de activos y
el retorno de la deuda del gobierno, es decir, el premio al capital o el
exceso de rendimiento \(r^i-r^j\), satisface
\(E[r^i]-E[r^j]=\theta Cov(r^i-r^j, g^c)\), de donde tenemos queel
coefiente de aversión relativa al riesgo es
\(\theta =\frac{E[r^i]-E[r^j]}{Cov(r^i-r^j, g^c)}\).

Para el análisis de corto plazo tenemos:

\begin{center}
\textbf{Cuadro 5.3. Coeficiente de aversión al riesgo 28 días}
\end{center}
\begin{table}[h!]
                \begin{center} 
                    \begin{tabular}{|c|c|c|c|} 
                        \hline 
                        \mbox{}\;\;\;\;\;\mbox{}& \; Notación\;  & \; Valor calculado\; \\ 
                        \hline 
                        \multirow{4}{*}{\begin{sideways}28 días\end{sideways}}  
                            &   &  \\
                        &$cov(r^i-r_{cp}^j, g^c)$  & $0.0009221631$ \\
                        &$E[r^i]-E[r^j_{cp}]$ & $0.07491655$ \\ 
                        &$\theta$ & $81.24002$ \\ 
                        &  &   \\
                        \hline \hline
                        \hline
                    \end{tabular}
                \end{center}
\end{table}

Para el análisis de largo plazo tenemos:

\begin{center}
\textbf{Cuadro 5.4. Coeficiente de aversión al riesgo 364 días}
\end{center}

\begin{table}[h!]
                \begin{center} 
                    \begin{tabular}{|c|c|c|c|} 
                        \hline 
                        \mbox{}\;\;\;\;\;\mbox{}& \; Notación\;  & \; Valor calculado\; \\ 
                        \hline 
                        \multirow{4}{*}{\begin{sideways}364 días\end{sideways}} 
                        &  &   \\
                        &$cov(r^i-r_{lp}^j, g^c)$  & $0.0004941631$ \\
                        &$E[r^i]-E[r^j_{lp}]$ & $0.07187854$ \\ 
                        &$\theta$ & $145.4551$ \\ 
                        &  &   \\
                        \hline \hline
                        \hline
                    \end{tabular}
                \end{center}
\end{table}

El coeficiente de aversión relativa al riesgo, es de 81.24 en el corto
plazo y de 145.45 en el largo plazo, que de acuerdo con Romer son
valores muy altos de aversión al riesgo. Esto implica que los individuos
preferirán aceptar una reducción de su consumo en un procentaje menor de
manera segura, que arriesgarse a una lotería en que un resultado sería
una reducción mucho mayor en el consumo.

\newpage

\hypertarget{ejercicio-5.g}{%
\subsubsection{Ejercicio 5.g}\label{ejercicio-5.g}}

\textbf{Ahora calcule la covarianza entre dicha diferencias y la tasa de
crecimiento real del consumo agregado DE BIENES IMPORTADOS {[}aquí hay
una serie: www.inegi.org.mx/temas/imcp/{]} de la economía mexicana.}

Para el corto plazo tenemos:

\begin{center}
\textbf{Cuadro 5.5. Estadística descriptiva 28 días - Consumo Importado -}
\end{center}

\begin{table}[h!]
                \begin{center} 
                    \begin{tabular}{|c|c|c|c|} 
                        \hline 
                        \mbox{}\;\;\;\;\;\mbox{}& \; Notación\;  & \; Valor calculado\; \\ 
                        \hline 
                        \multirow{4}{*}{\begin{sideways}28 días\end{sideways}}
                        &  &   \\
                        & $sd(g^c)$  & $0.1940291$  \\ 
                        &$sd(r^i-r_{cp}^j)$ & $0.2710058$ \\ 
                        &$corr(r^i-r_{cp}^j, g^c)$ & $-0.01814767$ \\ 
                        &$cov(r^i-r_{cp}^j, g^c)$  & $-0.0009542589$ \\                         
                        \hline \hline
                        \hline
                    \end{tabular}
                \end{center}
\end{table}

Para el largo plazo tenemos:

\begin{center}
\textbf{Cuadro 5.6. Estadística descriptiva 364 días - Consumo Importado -}
\end{center}

\begin{table}[h!]
                \begin{center} 
                    \begin{tabular}{|c|c|c|c|} 
                        \hline 
                        \mbox{}\;\;\;\;\;\mbox{}& \; Notación\;  & \; Valor calculado\; \\ 
                        \hline 
                        \multirow{4}{*}{\begin{sideways}364 días\end{sideways}}
                        &  &   \\
                        & $sd(g^c)$  & $0.1940291$  \\ 
                        &$sd(r^i-r_{lp}^j)$ & $0.2620966$ \\ 
                        &$corr(r^i-r_{lp}^j, g^c)$ & $-0.07331527$ \\ 
                        &$cov(r^i-r_{lp}^j, g^c)$  & $-0.003728401$ \\                      
                        \hline \hline
                        \hline
                    \end{tabular}
                \end{center}
\end{table}

\hypertarget{ejercicio-5.h}{%
\subsubsection{Ejercicio 5.h}\label{ejercicio-5.h}}

\textbf{Calcule el valor de aversión relativa al riesgo que implican
estos números, dado el supuesto de una utilidad con forma ARRC.}

En el corto plazo tenemos:

\begin{center}
\textbf{Cuadro 5.7. Coeficiente de aversión al riesgo 28 días - Consumo Importado -}
\end{center}

\begin{table}[h!]
                \begin{center} 
                    \begin{tabular}{|c|c|c|c|} 
                        \hline 
                        \mbox{}\;\;\;\;\;\mbox{}& \; Notación\;  & \; Valor calculado\; \\ 
                        \hline 
                        \multirow{4}{*}{\begin{sideways}28 días\end{sideways}}  
                            &   &  \\
                        &$cov(r^i-r_{cp}^j, g^c)$  & $-0.0009542589$ \\
                        &$E[r^i]-E[r^j_{cp}]$ & $0.0281719$ \\ 
                        &$\theta$ & $-29.52228$ \\ 
                        &  &   \\
                        \hline \hline
                        \hline
                    \end{tabular}
                \end{center}
\end{table}

En el largo plazo tenemos:

\begin{center}
\textbf{Cuadro 5.8. Coeficiente de aversión al riesgo 364 días - Consumo Importado -}
\end{center}

\begin{table}[h!]
                \begin{center} 
                    \begin{tabular}{|c|c|c|c|} 
                        \hline 
                        \mbox{}\;\;\;\;\;\mbox{}& \; Notación\;  & \; Valor calculado\; \\ 
                        \hline 
                        \multirow{4}{*}{\begin{sideways}364 días\end{sideways}}  
                            &   &  \\
                        &$cov(r^i-r_{lp}^j, g^c)$  & $-0.003728401$ \\
                        &$E[r^i]-E[r^j_{lp}]$ & $0.02561693$ \\ 
                        &$\theta$ & $-6.870755$ \\ 
                        &  &   \\
                        \hline \hline
                        \hline
                    \end{tabular}
                \end{center}
\end{table}

El coeficiente de aversión relativa al riesgo, es de -29.52 en el corto
plazo y de -6.87 en el largo plazo, lo que muestra un comportamiento de
amor al riesgo. Esto implica que los individuos preferirán arriesgarse a
una lotería en que un resultado sería una reducción mucho mayor en el
consumo que aceptar una reducción de su consumo en un procentaje menor
de manera segura.

\newpage

\hypertarget{ejercicio-6}{%
\subsection{Ejercicio 6}\label{ejercicio-6}}

\textbf{Utilice el método del árbol binomial para explicar el precio
P=80 de un activo y valuar un instrumento derivado tipo ``call'' sobre
él, con expiración un año después y precio de ejercicio K=P-N donde N es
el número de su equipo (!), asumiendo una tasa de interés de 5 por
ciento: {[}1 horas, 1 punto cada inciso{]}}

\hypertarget{ejercicio-6.a}{%
\subsubsection{Ejercicio 6.a}\label{ejercicio-6.a}}

\textbf{Haga supuestos sobre distintos valores que podría tomar el
activo un año después y sobre las probabilidades objetivas de que tome
esos valores. Infiera las probabilidades subjetivas de que el activo
tome dichos valores y de una explicación cualitativa sobre qué
circunstancias podrían generar la diferencia entre las probabilidades
objetivas y subjetivas observadas. (Si le salen probabilidades objetivas
y subjetivas iguales, favor de cambiar sus probabilidades objetivas por
otras. (!))}

A continuación se establecen los parámetros relevantes del problema
dados, o bien, supuestos:

\begin{itemize}
\item
  Precio del activo (P): \textbf{80}
\item
  Precio del ejercicio de la opción (K): \textbf{78}
\item
  Precios supuestos sobre el valor del activo en T=1(S): \textbf{85 y
  75}
\item
  Probabilidad supuesta de evento favorable / desfavorable:
  \(\mathbf{Q_{A} =0.65 / Q_{B} = 0.35}\)
\end{itemize}

Así pues, y con base en la teoría expuesta en Romer (2019) y en las
notas de clase, lo primero es establecer la lógica del precio de
ejercicio del activo, \(P_{0} = 78\). Este debe cumplirse de acuerdo a
que \(P_{0} = E^q[P_{1}]\).

Esto es, descomponiéndolo de acuerdo a dos probabilidades de eventos,
uno que afectará positivamente su valor, y el otro negativamente:
\(78 = \frac{(q_{A}(85) + q_{B}(75))}{R^{LR}}\) donde \(q_{i}\) son las
probabilidades subjetivas de cada evento, \(q_{A} + q_{B} = 1\) y
\(R^{LR} = 1 + r_{t,t+1}\). Es decir, el precio de ejercicio del activo
es igual al valor esperado del activo en T = 1, con sus correspondientes
probabilidades, traído a valor presente.

Dado que la tasa de interés es de 5\%, entonces \(R^{LR} = 1.05\) y
\(78(1.05) = (85 - 75)q_{A} + 75\). Ya que \(q_{B} = (1-q_{A})\),
despejando las probabilidades de ambos eventos obtenemos que:

\[q_{A} = 0.69\]

\[q_{B} = 0.31\] Que son las probabilidades subjetivas; recuérdese que
el supuesto de las probabilidades objetivas es:

\[ Q_{A} = 0.65\]

\[Q_{B} = 0.35\] Finalmente, supónganse que el activo en cuestión es un
acción de la empresa GameStop, dedicada a la renta y venta de
videojuegos en formato físico. Derivado de la creciente digitalización
de estos bienes, dicha empresa ha sufrido pérdidas enormes en sus
ingresos durante los últimos 10 años, llevando el valor de su acción
hasta un precio de \$80 MXN. Sin embargo, usuarios de la red social
Reddit se han organizado y decidido realizar una compra masiva de
acciones de GameStop, con el objetivo de aumentar la demanda de éstas y,
además, provocar pérdidas millonarias a las empresas que se han
beneficiado durante años de la decadencia de GameStop mediante las
llamadas \emph{short-calls}.

Por tal motivo, un individuo decide adquirir opciones de compra
\emph{call} de las acciones de GameStop con expiración de un año,
basándose en los análisis profesionales que diferentes empresas han
realizado, donde aseguran que el fenómeno de Reddit continuará
indefinidamente y las acciones de la empresa, con probabilidad de 65\%,
valdrán 85 pesos en un año: en caso contrario, con probabilidad de 35\%,
el fenómeno se extinguirá y el precio de las acciones de GameStop
continuará decayendo, con valor de 75 pesos en un año.

Intuitivamente, la diferencia entre las probabilidades objetiva y
subjetiva recae en que la primera se basa en observaciones registradas
por autoridades competentes sobre un fenómeno con un historial
informativo amplio; por su parte, la probabilidad subjetiva puede tomar
en cuenta intuiciones, experiencias propias y expectativas sobre un
fenómeno. En este ejemplo, la probabilidad objetiva del evento favorable
es más baja debido a que las instancias que la calculan podrían estar
incurriendo en conflictos de interés, toda vez que sus preferencias
bursátiles pueden estar alineadas a aquellas empresas que han sufrido
durante el fenómeno Reddit, o bien, por presiones económicas y
políticas.

\hypertarget{ejercicio-6.b}{%
\subsubsection{Ejercicio 6.b}\label{ejercicio-6.b}}

\textbf{Calcule el valor de los pagos de la opción ante los distintos
escenarios futuros y con ellos calcule el precio actual del instrumento
derivado.}

Con base en lo anterior, el precio que la opción de \emph{call} debería
tomar después de un año, de acuerdo a las probabilidades subjetivas, es:

\[O_{S} = \frac{(q_{A}(85 - 78) + q_{B}(0))}{R^{LR}} = 4.6\]

Y de acuerdo a las probabilidades objetivas:

\[O_{O} = \frac{(Q_{A}(85 - 78) + Q_{B}(0))}{R^{LR}} = 4.33\bar{3}\]
Suponiendo que \(O_{O}\) es el valor observable de la opción, puede
verse que el valor que debería tener, de acuerdo a las probabilidades
subjetivas, es ligeramente mayor, por lo que su compra es sensata.
Finalmente, es evidente que \(q_{A} > Q_{A}\) y que \(q_{B} < Q_{B}\),
esto es: los retornos esperados del activo en un evento favorable son
muy valorados el día de hoy y, asimismo, los retornos del activo en caso
de un evento desfavorable son poco valorados en el presente.

\end{document}
